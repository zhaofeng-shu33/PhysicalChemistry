
\documentclass{article}
%%%%%%%%%%%%%%%%%%%%%%%%%%%%%%%%%%%%%%%%%%%%%%%%%%%%%%%%%%%%%%%%%%%%%%%%%%%%%%%%%%%%%%%%%%%%%%%%%%%%%%%%%%%%%%%%%%%%%%%%%%%%%%%%%%%%%%%%%%%%%%%%%%%%%%%%%%%%%%%%%%%%%%%%%%%%%%%%%%%%%%%%%%%%%%%%%%%%%%%%%%%%%%%%%%%%%%%%%%%%%%%%%%%%%%%%%%%%%%%%%%%%%%%%%%%%
\usepackage{amsmath}

\setcounter{MaxMatrixCols}{10}
%TCIDATA{OutputFilter=LATEX.DLL}
%TCIDATA{Version=5.00.0.2552}
%TCIDATA{<META NAME="SaveForMode" CONTENT="1">}
%TCIDATA{Created=Thursday, September 17, 2015 22:03:22}
%TCIDATA{LastRevised=Sunday, November 22, 2015 22:19:40}
%TCIDATA{<META NAME="GraphicsSave" CONTENT="32">}
%TCIDATA{<META NAME="DocumentShell" CONTENT="Standard LaTeX\Blank - Standard LaTeX Article">}
%TCIDATA{CSTFile=40 LaTeX article.cst}

\newtheorem{theorem}{Theorem}
\newtheorem{acknowledgement}[theorem]{Acknowledgement}
\newtheorem{algorithm}[theorem]{Algorithm}
\newtheorem{axiom}[theorem]{Axiom}
\newtheorem{case}[theorem]{Case}
\newtheorem{claim}[theorem]{Claim}
\newtheorem{conclusion}[theorem]{Conclusion}
\newtheorem{condition}[theorem]{Condition}
\newtheorem{conjecture}[theorem]{Conjecture}
\newtheorem{corollary}[theorem]{Corollary}
\newtheorem{criterion}[theorem]{Criterion}
\newtheorem{definition}[theorem]{Definition}
\newtheorem{example}[theorem]{Example}
\newtheorem{exercise}[theorem]{Exercise}
\newtheorem{lemma}[theorem]{Lemma}
\newtheorem{notation}[theorem]{Notation}
\newtheorem{problem}[theorem]{Problem}
\newtheorem{proposition}[theorem]{Proposition}
\newtheorem{remark}[theorem]{Remark}
\newtheorem{solution}[theorem]{Solution}
\newtheorem{summary}[theorem]{Summary}
\newenvironment{proof}[1][Proof]{\noindent\textbf{#1.} }{\ \rule{0.5em}{0.5em}}
\input{tcilatex}

\begin{document}


Using Van der Waals Equation to calculate the Boyle temperature of the
hydrogen gas.

From pages 9-10 of the textbook, the Boyle temperature $T_{B}$ satisfies $%
\left[ \frac{\partial (pV_{m})}{\partial p}\right] _{T_{B}}\rightarrow 0$ as 
$p\rightarrow 0.$

From the Van der Waals Equation: $(p-\frac{a}{V_{m}^{2}})(V_{m}-b)=RT,$ keep 
$T=const$ and let $y=PV_{m},$then \qquad\ substitute $V_{m}=\frac{y}{P}$
into the equation to cancel out $V_{m},$ after rearranging we can get $%
y^{3}-RTy^{2}-abp^{2}=p(by^{2}-ay).y$ can be treated as a function with
single variable $p$.Take the derivative about $p$ on both sides of the above
equation, and then let $p\rightarrow 0,$ from the condition we have $%
y^{\prime }\rightarrow 0$ and the left hand tends to zero. We therefore get $%
by^{2}-ay=0$ (1) If we take the limit $p\rightarrow 0$ directly in the
derived equation,we can get $y^{3}-RTy^{2}=0$ (2) From (1)(2) we can solve
out $\underset{p\rightarrow 0}{\lim }y=\frac{a}{b}=RT$ (3) since $\underset{%
p\rightarrow 0}{\lim }y=0$ is impossible, because that means $\underset{%
p\rightarrow 0}{\lim }V_{m}=\infty .$At last, let the const $T$ be the Boyle
temperature $T_{B},$ it follows from (3) that $T_{B}=\frac{a}{bR}.$For
hydrogen gas substitute $a=0.0274,b=2.66\times 10^{-5},R=8.314$ (the unit is
omitted here for convenience), we get $T_{B}\approx 123.90K.$

\end{document}
