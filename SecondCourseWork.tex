
\documentclass{article}
%%%%%%%%%%%%%%%%%%%%%%%%%%%%%%%%%%%%%%%%%%%%%%%%%%%%%%%%%%%%%%%%%%%%%%%%%%%%%%%%%%%%%%%%%%%%%%%%%%%%%%%%%%%%%%%%%%%%%%%%%%%%%%%%%%%%%%%%%%%%%%%%%%%%%%%%%%%%%%%%%%%%%%%%%%%%%%%%%%%%%%%%%%%%%%%%%%%%%%%%%%%%%%%%%%%%%%%%%%%%%%%%%%%%%%%%%%%%%%%%%%%%%%%%%%%%
\usepackage{amsmath}

\setcounter{MaxMatrixCols}{10}
%TCIDATA{OutputFilter=LATEX.DLL}
%TCIDATA{Version=5.00.0.2552}
%TCIDATA{<META NAME="SaveForMode" CONTENT="1">}
%TCIDATA{Created=Thursday, September 24, 2015 18:11:16}
%TCIDATA{LastRevised=Thursday, October 15, 2015 22:57:08}
%TCIDATA{<META NAME="GraphicsSave" CONTENT="32">}
%TCIDATA{<META NAME="DocumentShell" CONTENT="Scientific Notebook\Blank Document">}
%TCIDATA{CSTFile=Math with theorems suppressed.cst}
%TCIDATA{PageSetup=72,72,72,72,0}
%TCIDATA{AllPages=
%F=36,\PARA{038<p type="texpara" tag="Body Text" >\hfill \thepage}
%}


\newtheorem{theorem}{Theorem}
\newtheorem{acknowledgement}[theorem]{Acknowledgement}
\newtheorem{algorithm}[theorem]{Algorithm}
\newtheorem{axiom}[theorem]{Axiom}
\newtheorem{case}[theorem]{Case}
\newtheorem{claim}[theorem]{Claim}
\newtheorem{conclusion}[theorem]{Conclusion}
\newtheorem{condition}[theorem]{Condition}
\newtheorem{conjecture}[theorem]{Conjecture}
\newtheorem{corollary}[theorem]{Corollary}
\newtheorem{criterion}[theorem]{Criterion}
\newtheorem{definition}[theorem]{Definition}
\newtheorem{example}[theorem]{Example}
\newtheorem{exercise}[theorem]{Exercise}
\newtheorem{lemma}[theorem]{Lemma}
\newtheorem{notation}[theorem]{Notation}
\newtheorem{problem}[theorem]{Problem}
\newtheorem{proposition}[theorem]{Proposition}
\newtheorem{remark}[theorem]{Remark}
\newtheorem{solution}[theorem]{Solution}
\newtheorem{summary}[theorem]{Summary}
\newenvironment{proof}[1][Proof]{\noindent\textbf{#1.} }{\ \rule{0.5em}{0.5em}}
\input{tcilatex}

\begin{document}


\bigskip \U{7269}\U{5316}\U{7b2c}\U{4e8c}\U{5468}\U{4f5c}\U{4e1a}\qquad
\qquad \qquad \U{8d75}\U{4e30}2013012178

Problem 1-10

$\left( 1\right) $\U{5bf9}\U{4e8e}N$_{2},$\U{7531}\U{7406}\U{60f3}\U{6c14}%
\U{4f53}\U{72b6}\U{6001}\U{65b9}\U{7a0b}$\qquad P=\frac{nRT}{V}=\frac{1\unit{%
mol}\times 8.314\unit{J}\cdot \unit{mol}^{-1}\cdot \unit{K}^{-1}\times 273.2%
\unit{K}}{70.3\times 10^{-6}\unit{m}^{3}}=\allowbreak 3.\,\allowbreak
231\,0\times 10^{4}\unit{kPa}$

$\left( 2\right) $\U{82e5}\U{6539}\U{7528}Van der Waals' equation, \U{67e5}%
\U{8868}\U{77e5}\qquad $a=0.141\unit{Pa}\cdot \unit{m}^{6}\cdot \unit{mol}%
^{-2},b=3.91\times 10^{-5}\unit{m}^{3}\cdot \unit{mol}^{-1}.$

$P=\frac{RT}{V_{m}-b}-\frac{a}{V_{m}^{2}}=\frac{8.314\unit{J}\cdot \unit{mol}%
^{-1}\cdot \unit{K}^{-1}\times 273.2\unit{K}}{70.3\times 10^{-6}\unit{m}%
^{3}\cdot \unit{mol}^{-1}-3.91\times 10^{-5}\unit{m}^{3}\cdot \unit{mol}^{-1}%
}-\frac{0.141\unit{Pa}\cdot \unit{m}^{6}\cdot \unit{mol}^{-2}}{\left(
70.3\times 10^{-6}\unit{m}^{3}\cdot \unit{mol}^{-1}\right) ^{2}}=\allowbreak
4.\,\allowbreak 427\times 10^{4}\unit{kPa}$

$\left( 3\right) $\U{82e5}\U{7528}\U{538b}\U{7f29}\U{56e0}\U{5b50}\U{56fe}%
\U{7684}\U{65b9}\U{6cd5},\U{5728}\U{7ed9}\U{5b9a}\U{7684}\U{6e29}\U{5ea6}$%
T=273.2\unit{K}$\U{548c}\U{4f53}\U{79ef}$V_{m}=70.3\times 10^{-6}\unit{m}%
^{3}/\unit{mol}$\U{7684}\U{6761}\U{4ef6}\U{4e0b}

\U{67e5}\U{8868}\U{5f97}$N_{2}$\U{7684}\U{4e34}\U{754c}\U{53c2}\U{6570}%
\U{4e3a}$T_{c}=126.1\unit{K},P_{c}=3.39\unit{MPa}.$

\U{538b}\U{7f29}\U{56e0}\U{5b50}$Z=\frac{PV_{m}}{RT}=\frac{P_{r}P_{c}V_{m}}{%
RT}=\frac{3.39\unit{MPa}\times 70.3\times 10^{-6}\unit{m}^{3}/\unit{mol}}{%
8.314\unit{J}\cdot \unit{mol}^{-1}\cdot \unit{K}^{-1}\times 273.2\unit{K}}%
P_{r}\approx \allowbreak 0.105P_{r}.$

\U{53e6}\U{4e00}\U{65b9}\U{9762}$N_{2}$\U{7684}\U{72b6}\U{6001}\U{8981}%
\U{843d}\U{5728}$T_{r}=\frac{T}{T_{c}}=\frac{273.2\unit{K}}{126.1\unit{K}}%
\approx 2.\,\allowbreak 17$\U{7684}\U{7b49}\U{5bf9}\U{6bd4}\U{6e29}\U{5ea6}%
\U{7ebf}\U{4e0a}$\U{3002} $

\U{5728}$Z-P_{r}$\U{56fe}\U{4e0a}\U{753b}\U{51fa}\U{76f4}\U{7ebf}$%
\allowbreak Z=0.105P_{r}$ \U{6c42}\U{51fa}\U{5176}\U{4e0e}\U{6b64}\U{7b49}%
\U{5bf9}\U{6bd4}\U{6e29}\U{5ea6}\U{7ebf}\U{7684}\U{4ea4}\U{70b9}\U{5373}%
\U{4e3a}$N_{2}$\U{7684}\U{72b6}\U{6001},\U{5982}\U{4e0b}\U{56fe}\U{6240}%
\U{793a}$:$

\FRAME{ftbpF}{4.6942in}{1.2825in}{0pt}{}{}{Figure}{\special{language
"Scientific Word";type "GRAPHIC";display "USEDEF";valid_file "T";width
4.6942in;height 1.2825in;depth 0pt;original-width 8.1042in;original-height
2.9482in;cropleft "0";croptop "1";cropright "1";cropbottom "0";tempfilename
'NV6VBQ0B.wmf';tempfile-properties "XPR";}}

\U{56fe}\U{4e2d}\U{753b}\U{51fa}\U{4e86}\U{76f4}\U{7ebf}$\allowbreak
Z=0.105P_{r}$\U{4e0e}$T_{r}=2$\U{7684}\U{7b49}\U{5bf9}\U{6bd4}\U{6e29}%
\U{5ea6}\U{7ebf}\U{7684}\U{4ea4}\U{70b9},\U{5bf9}\U{5e94}\U{7684}\U{5bf9}%
\U{6bd4}\U{538b}\U{529b}\U{5927}\U{7ea6}\U{4e3a}$12.5\U{3002} $

\U{56e0}\U{6b64}$P=P_{c}P_{r}=3.39\unit{MPa}\times 12.5=\allowbreak
42.\,\allowbreak 375\unit{MPa}\approx 4.24\times 10^{4}\unit{kPa}.$

\U{6bd4}\U{8f83}$\left( 1,2,3\right) $\U{4e2d}\U{6c42}\U{51fa}\U{7684}%
\U{7ed3}\U{679c}\U{53ef}\U{77e5}$\left( 3\right) $\U{4e0e}\U{5b9e}\U{9645}%
\U{6d4b}\U{91cf}\U{503c}\U{6700}\U{4e3a}\U{63a5}\U{8fd1},\U{7528}Van der
Waal's equation\U{5f97}\U{5230}\U{7684}\U{7ed3}\U{679c}\U{6709}\U{4e00}%
\U{5b9a}\U{504f}\U{5dee}$,$\U{800c}\U{7531}\U{4e8e}\U{5916}\U{538b}\U{5f88}%
\U{5927}$,$\U{7528}\U{7406}\U{60f3}\U{6c14}\U{4f53}\U{72b6}\U{6001}\U{65b9}%
\U{7a0b}\U{6c42}\U{51fa}\U{7684}\U{7ed3}\U{679c}\U{5219}\U{5b8c}\U{5168}%
\U{504f}\U{79bb}\U{771f}\U{5b9e}\U{503c}$\U{3002} $

Problem 2-1 (1) \U{7531}\U{70ed}\U{529b}\U{5b66}\U{7b2c}\U{4e00}\U{5b9a}%
\U{5f8b}:$\Delta U=Q-W=397.5\unit{J}+167.4\unit{J}=Q-83.68\unit{J};$

\U{89e3}\U{5f97} $Q=\allowbreak 481.\,\allowbreak 22\unit{J}.$\U{5373}%
\U{653e}\U{70ed}$\allowbreak 481.\,\allowbreak 22\unit{J}.$

$-\Delta U=Q^{\prime }-125.5\unit{J}$

\U{89e3}\U{5f97} $Q^{\prime }=-439.\,\allowbreak 4\unit{J},$\U{5373}\U{5438}%
\U{70ed}$439.\,\allowbreak 4\unit{J}.$

Problem 2-5

$W=\int_{V_{1}}^{V_{2}}PdV=\int_{V_{1}}^{V_{2}}\left( \frac{nRT}{V-nb}-\frac{%
n^{2}a}{V^{2}}\right) dV=nRT\ln \left( \frac{V_{2}-b}{V_{1}-b}\right)
+n^{2}a\left( \frac{1}{V_{2}}-\frac{1}{V_{1}}\right) ,$

\U{4ee3}\U{5165}\U{6570}\U{636e} $T=300\unit{K},V_{1}=0.010\unit{m}%
^{3},V_{2}=0.030\unit{m}^{3},$

$a=0.5563\unit{Pa}\cdot \unit{m}^{6}\cdot \unit{mol}^{-2},b=8.4\times 10^{-5}%
\unit{m}^{3}\cdot \unit{mol}^{-1}$

$W=1\unit{mol}\times 8.314\unit{J}\cdot \unit{mol}^{-1}\cdot \unit{K}%
^{-1}\times 300\unit{K}\times \ln \left( \frac{0.030-8.4\times 10^{-5}}{%
0.010-8.4\times 10^{-5}}\right) $

$+\left( 1\unit{mol}\right) ^{2}\times 0.5563\unit{Pa}\cdot \unit{m}%
^{6}\cdot \unit{mol}^{-2}\times \left( \frac{1}{0.030\unit{m}^{3}\cdot \unit{%
mol}^{-1}}-\frac{1}{0.010\unit{m}^{3}\cdot \unit{mol}^{-1}}\right)
=\allowbreak 2717.\,\allowbreak 1\unit{J}$

Problem 2-6

$\left( 1\right) $\U{5faa}\U{73af}\U{8fc7}\U{7a0b}\U{7cfb}\U{7edf}\U{6240}%
\U{505a}\U{7684}\U{529f}\U{4e3a}\U{66f2}\U{8fb9}\U{4e09}\U{89d2}\U{5f62}ABC%
\U{6240}\U{56f4}\U{7684}\U{9762}\U{79ef}

$\left( 2\right) B\rightarrow C$\U{7684}$\Delta
U=U_{C}-U_{B}=U_{C}-U_{A}=\Delta U_{A\rightarrow C}=-W\qquad $\U{5373}%
\U{4e3a}\U{66f2}\U{7ebf}AC\U{4e0b}\U{7684}\U{9762}\U{79ef}\U{7684}\U{8d1f}%
\U{503c}

$\left( 3\right) \Delta U_{B\rightarrow C}=Q-W^{\prime }\qquad Q=W^{\prime
}-W$,\qquad \U{5373}\U{4e3a}BC\U{7ebf}\U{6bb5}\U{4e0b}\U{7684}\U{9762}%
\U{79ef}\U{4e0e}\U{66f2}\U{7ebf}AC\U{4e0b}\U{7684}\U{9762}\U{79ef}\U{7684}%
\U{5dee}\U{503c}.

Added Problem From Atkins' Physical Chemistry:

Calculate the volume occupied by 1.00 mol N$_{2}$ using the van der Waals
equations in the form of a virial

expansion at (a) its critical temperature, (b) its Boyle temperature. Assume
that the pressure is 10atm throughtout. At what temperature is the gas most
perfect? Use the following data:$T_{c}=126.3\unit{K},$

$a=1.408\unit{atm}\cdot \unit{l}^{2}\cdot \unit{mol}^{-2},b=0.0391\unit{l}%
\cdot \unit{mol}^{-1}.$

Solution: $P=\frac{RT}{V_{m}-b}-\frac{a}{V_{m}^{2}}=\frac{RT}{V_{m}}\frac{1}{%
1-\frac{b}{V_{m}}}-\frac{a}{V_{m}^{2}}\overset{Taylor\_Expansion}{=}\frac{RT%
}{V_{m}}\left( 1+\frac{b}{V_{m}}\right) -\frac{a}{V_{m}^{2}}$

=$\frac{RT}{V_{m}}+\frac{RTb-a}{V_{m}^{2}}.\implies PV_{m}=RT+\frac{RTb-a}{%
V_{m}},$ which is in the form of virial expansion.

$\left( a\right) $ at critical temperature $T_{c}=126.3\unit{K},$after
substituting given data into the above equation we can get

a quadratic equation about $V_{m}:$

$10\unit{atm}\times V_{m}^{2}=8.206\times 10^{-2}\unit{l}\cdot \unit{atm}%
\cdot \unit{K}^{-1}\cdot \unit{mol}^{-1}\times 126.3\unit{K}\times
V_{m}+C=\allowbreak 10.\,\allowbreak 364\left( \unit{l}\right) \frac{\unit{%
atm}}{\unit{mol}}\times V_{m}+C,$

where $C=8.206\times 10^{-2}\unit{l}\cdot \unit{atm}\cdot \unit{K}^{-1}\cdot 
\unit{mol}^{-1}\times 126.3\unit{K}\times 0.0391\unit{l}\cdot \unit{mol}%
^{-1}-1.408\unit{atm}\cdot \unit{l}^{2}\cdot \unit{mol}^{-2}$

$=-1.\,\allowbreak 002\,8\unit{l}^{2}\frac{\unit{atm}}{\unit{mol}^{2}}.$

After cancelling out the units, the equation reduces to $%
10V_{m}^{2}-10.364V_{m}+1.0028=0;$

The acceptable root is $V_{m}=0.928\unit{l}\cdot \unit{mol}^{-1}\left( \text{%
another real root is too small and impossible}\right) $

$\left( b\right) $ At Boyle temperature, the first coefficient in the virial
expansion vanishes

$\implies RTb-a=0$

$\implies T=\frac{a}{bR}=\frac{1.408\unit{atm}\cdot \unit{l}^{2}\cdot \unit{%
mol}^{-2}}{0.0391\unit{l}\cdot \unit{mol}^{-1}\times 8.206\times 10^{-2}%
\unit{l}\cdot \unit{atm}\cdot \unit{K}^{-1}\cdot \unit{mol}^{-1}}%
=\allowbreak 438.\,\allowbreak 83\unit{K}.$

Substituting the solved temperature into the approximate equation:$PV_{m}=RT$
gives

$V_{m}=\frac{RT}{P}=\frac{8.206\times 10^{-2}\unit{l}\cdot \unit{atm}\cdot 
\unit{K}^{-1}\cdot \unit{mol}^{-1}\times \allowbreak 438.\,\allowbreak 83%
\unit{K}}{10\unit{atm}}=\allowbreak 3.\,\allowbreak 601\frac{\unit{l}}{\unit{%
mol}},$which is much larger than the results in $\left( a\right) .$

The larger $V_{m}$ in $\left( b\right) $ shows that the same amount of gas
molecules occupy larger volume in the container and thus they have less
intermolecular interaction, which implies that their behavior is much

more like the ideal gas. $\left( \text{the approximate equation }PV_{m}=RT%
\text{ also implies this conclusion.}\right) $

\end{document}
