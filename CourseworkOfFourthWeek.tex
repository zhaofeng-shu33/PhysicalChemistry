
\documentclass{article}
%%%%%%%%%%%%%%%%%%%%%%%%%%%%%%%%%%%%%%%%%%%%%%%%%%%%%%%%%%%%%%%%%%%%%%%%%%%%%%%%%%%%%%%%%%%%%%%%%%%%%%%%%%%%%%%%%%%%%%%%%%%%%%%%%%%%%%%%%%%%%%%%%%%%%%%%%%%%%%%%%%%%%%%%%%%%%%%%%%%%%%%%%%%%%%%%%%%%%%%%%%%%%%%%%%%%%%%%%%%%%%%%%%%%%%%%%%%%%%%%%%%%%%%%%%%%
\usepackage{amssymb}
\usepackage{amsmath}

\setcounter{MaxMatrixCols}{10}
%TCIDATA{OutputFilter=LATEX.DLL}
%TCIDATA{Version=5.00.0.2552}
%TCIDATA{<META NAME="SaveForMode" CONTENT="1">}
%TCIDATA{Created=Tuesday, October 06, 2015 17:47:35}
%TCIDATA{LastRevised=Sunday, October 11, 2015 13:30:04}
%TCIDATA{<META NAME="GraphicsSave" CONTENT="32">}
%TCIDATA{<META NAME="DocumentShell" CONTENT="Scientific Notebook\Blank Document">}
%TCIDATA{CSTFile=Math with theorems suppressed.cst}
%TCIDATA{PageSetup=72,72,72,72,0}
%TCIDATA{AllPages=
%F=36,\PARA{038<p type="texpara" tag="Body Text" >\hfill \thepage}
%}


\newtheorem{theorem}{Theorem}
\newtheorem{acknowledgement}[theorem]{Acknowledgement}
\newtheorem{algorithm}[theorem]{Algorithm}
\newtheorem{axiom}[theorem]{Axiom}
\newtheorem{case}[theorem]{Case}
\newtheorem{claim}[theorem]{Claim}
\newtheorem{conclusion}[theorem]{Conclusion}
\newtheorem{condition}[theorem]{Condition}
\newtheorem{conjecture}[theorem]{Conjecture}
\newtheorem{corollary}[theorem]{Corollary}
\newtheorem{criterion}[theorem]{Criterion}
\newtheorem{definition}[theorem]{Definition}
\newtheorem{example}[theorem]{Example}
\newtheorem{exercise}[theorem]{Exercise}
\newtheorem{lemma}[theorem]{Lemma}
\newtheorem{notation}[theorem]{Notation}
\newtheorem{problem}[theorem]{Problem}
\newtheorem{proposition}[theorem]{Proposition}
\newtheorem{remark}[theorem]{Remark}
\newtheorem{solution}[theorem]{Solution}
\newtheorem{summary}[theorem]{Summary}
\newenvironment{proof}[1][Proof]{\noindent\textbf{#1.} }{\ \rule{0.5em}{0.5em}}
\input{tcilatex}

\begin{document}


\bigskip 物化第四周作业\qquad 
赵丰\qquad 2013012178

2-3$\left( 1\right) $为计算蒸发的过%
程热,该过程可看作等%
压条件下不做非体积%
功的相变过程,考虑到%
外压始终为$101325\unit{Pa},$因此%
对于蒸发过程的体积%
功,

$\Delta V=V_{2}-V_{1}=18\times 10^{-3}\unit{kg}\times (1.667\unit{m}^{3}/%
\unit{kg}-1.043\times 10^{-3}\unit{m}^{3}/\unit{kg})=\allowbreak
2.\,\allowbreak 998\,7\times 10^{-2}\unit{m}^{3}$

$W=P_{ex}\Delta V=101325\unit{Pa}\times \allowbreak \allowbreak
2.\,\allowbreak 998\,7\times 10^{-2}\unit{m}^{3}=\allowbreak
3038.\,\allowbreak 4\unit{J}.$此即环境所%
得到的功.

\bigskip $\left( 2\right) $若液态水的体%
积忽略不计,$W^{\prime }=P_{ex}V_{gas}=101325%
\unit{Pa}\times (\allowbreak 18\times 10^{-3}\unit{kg}\times 1.667\unit{m}%
^{3}/\unit{kg})=\allowbreak 3040.\,\allowbreak 4\unit{J}.$

若$\left( 1\right) $中计算得到的%
结果$W$为真实值$,$那么$%
W^{\prime }$对于$W$的相对误差%
为\qquad

$e=\frac{\left\vert W^{\prime }-W\right\vert }{\left\vert W\right\vert }=%
\frac{\allowbreak 3040.4\unit{J}-\allowbreak 3038.\,\allowbreak 4\unit{J}}{%
\allowbreak 3038.\,\allowbreak 4\unit{J}}=\allowbreak 6.\,\allowbreak
582\,4\times 10^{-4}.$

相对误差$e$很小,说明液%
态水的体积忽略对计%
算结果影响不大$,$这种%
近似是合理的.

$\left( 3\right) $若将水蒸气视为%
理气$,W^{\prime \prime }=P_{ex}\Delta V=P_{\text{末}}V_{%
\text{末}}=nRT$

$=1\unit{mol}\times 8.314\unit{J}/\left( \unit{mol}\cdot \unit{K}\right)
\times 373\unit{K}=\allowbreak 3101.\,\allowbreak 1\unit{J}.$

可以看出$\left( 3\right) $与$\left( 1\right) $%
的结果近似相等$,$这不%
仅说明将水蒸气视为%
理气是合理的$,$且说明%
借用物态方程来计算%
功将与直接计算结果%
吻合$.$

$\left( 4\right) $该过程可看作等%
压条件下不做非体积%
功的相变过程$,$因此反%
应热等于焓变,焓变为%
状态量,只与初末系统%
的初末状态有关,反应%
焓变即为水蒸气在正%
常沸点的蒸发焓其在%
数值上等于相应的摩%
尔汽化热乘以物质的%
量\qquad $\Delta H=C_{p}\times 1\unit{mol}=40.63\unit{kJ}\cdot \unit{%
mol}^{-1}\times 1\unit{mol}=\allowbreak 40.\,\allowbreak 63\unit{kJ}%
.\therefore Q=\Delta H=\allowbreak 40.\,\allowbreak 63\unit{kJ}$

由热力学第一定律\qquad $%
\Delta U=Q-W=\allowbreak 40.\,\allowbreak 63\unit{kJ}-\allowbreak
3038.\,\allowbreak 4\unit{J}=\allowbreak 37592\unit{J}.$

$\left( 5\right) \Delta H>W\impliedby \Delta U>0\qquad $因为%
水蒸气汽化是能量增%
加的过程$:\Delta U>0,$所以汽%
化热大于系统所作的%
功$W.$

\bigskip 2-8 $\left( 1,2\right) $对氢气,因其%
为双原子气体分子,其%
heat capacity ratio为$\gamma =\frac{\frac{7}{2}}{\frac{5}{2}}=\frac{7}{%
5}$,由过程方程得$T_{1}^{\gamma
}P_{1}^{1-\gamma }=T_{2}^{\gamma }P_{2}^{1-\gamma }\qquad \left( \ast
1\right) $

又由末态的物态方程%
知\qquad $P_{2}V_{2}=nRT_{2}\qquad \left( \ast 2\right) $

由$\left( \ast 1,\ast 2\right) ,$代入已知%
数据,可解得\qquad $P_{2}=935837\unit{Pa}%
,\qquad T_{2}=562.8\unit{K}$

$\left( 3\right) $因气体被压缩,环%
境做正功,其值为$%
W=-\int_{V_{1}}^{V_{2}}PdV,$

由过程方程 $PV^{\gamma }=P_{1}V_{1}^{\gamma
}=Const$ 解出$P$代入上式得

$W=-\int_{V_{1}}^{V_{2}}\frac{P_{1}V_{1}^{\gamma }}{V^{\gamma }}dV=P_{1}%
\frac{V_{1}-V_{1}^{\gamma }V_{2}^{1-\gamma }}{1-\gamma }$由过%
程方程和物态方程进%
一步化简得$W=\frac{nR\left( T_{1}-T_{2}\right) 
}{1-\gamma }\approx 5499.9\unit{J}.$

2-19 Show that $C_{P}-C_{V}=-\left( \frac{\partial P}{\partial T}\right) _{V}%
\left[ \left( \frac{\partial H}{\partial P}\right) _{T}-V\right] .$

Proof: $C_{P}-C_{V}=\left( \frac{\partial H}{\partial T}\right) _{P}-\left( 
\frac{\partial U}{\partial T}\right) _{V}=\left( \frac{\partial H}{\partial T%
}\right) _{P}-\left( \frac{\partial \left( H-PV\right) }{\partial T}\right)
_{V}$

$=\left( \frac{\partial H}{\partial T}\right) _{P}-\left( \frac{\partial H}{%
\partial T}\right) _{V}+V\left( \frac{\partial P}{\partial T}\right)
_{V}\qquad \left( \ast \right) $

Since $dH=\left( \frac{\partial H}{\partial T}\right) _{P}dT+\left( \frac{%
\partial H}{\partial P}\right) _{T}dP,$keeping V at constant and dividing
this total differential by $dT$

$\implies \left( \frac{\partial H}{\partial T}\right) _{V}=\left( \frac{%
\partial H}{\partial T}\right) _{P}+\left( \frac{\partial H}{\partial P}%
\right) _{T}\left( \frac{\partial P}{\partial T}\right) _{V},$and
substituting the expression of $\left( \frac{\partial H}{\partial T}\right)
_{V}$ into $\left( \ast \right) $

$\implies $ $C_{P}-C_{V}=-\left( \frac{\partial H}{\partial P}\right)
_{T}\left( \frac{\partial P}{\partial T}\right) _{V}+V\left( \frac{\partial P%
}{\partial T}\right) _{V}=-\left( \frac{\partial P}{\partial T}\right) _{V}%
\left[ \left( \frac{\partial H}{\partial P}\right) _{T}-V\right] .$

2-20 由$2$-3题的结论$,\Delta H=a\unit{J}%
,\Delta U=Q-W=$ $\Delta H-W=\left( a-3101.1\right) \unit{J},$其中%
$W$在2-3$\left( 3\right) $中已算出$.$

$100\times 75.29\times (-1)+x+\int_{273}^{373}(30+10.71\times
10^{-3}y)dy=40.66\times 10^{3}$, Solution is: $\left\{ \left[ x=44843.\right]
\right\} $

\end{document}
