
\documentclass{ctexart}
%%%%%%%%%%%%%%%%%%%%%%%%%%%%%%%%%%%%%%%%%%%%%%%%%%%%%%%%%%%%%%%%%%%%%%%%%%%%%%%%%%%%%%%%%%%%%%%%%%%%%%%%%%%%%%%%%%%%%%%%%%%%%%%%%%%%%%%%%%%%%%%%%%%%%%%%%%%%%%%%%%%%%%%%%%%%%%%%%%%%%%%%%%%%%%%%%%%%%%%%%%%%%%%%%%%%%%%%%%%%%%%%%%%%%%%%%%%%%%%%%%%%%%%%%%%%
\usepackage{amsmath}

\setcounter{MaxMatrixCols}{10}
%TCIDATA{OutputFilter=LATEX.DLL}
%TCIDATA{Version=5.00.0.2552}
%TCIDATA{<META NAME="SaveForMode" CONTENT="1">}
%TCIDATA{Created=Thursday, November 05, 2015 19:43:33}
%TCIDATA{LastRevised=Saturday, November 07, 2015 13:05:53}
%TCIDATA{<META NAME="GraphicsSave" CONTENT="32">}
%TCIDATA{<META NAME="DocumentShell" CONTENT="Scientific Notebook\Blank Document">}
%TCIDATA{CSTFile=Math with theorems suppressed.cst}
%TCIDATA{PageSetup=72,72,72,72,0}
%TCIDATA{AllPages=
%F=36,\PARA{038<p type="texpara" tag="Body Text" >\hfill \thepage}
%}


\newtheorem{theorem}{Theorem}
\newtheorem{acknowledgement}[theorem]{Acknowledgement}
\newtheorem{algorithm}[theorem]{Algorithm}
\newtheorem{axiom}[theorem]{Axiom}
\newtheorem{case}[theorem]{Case}
\newtheorem{claim}[theorem]{Claim}
\newtheorem{conclusion}[theorem]{Conclusion}
\newtheorem{condition}[theorem]{Condition}
\newtheorem{conjecture}[theorem]{Conjecture}
\newtheorem{corollary}[theorem]{Corollary}
\newtheorem{criterion}[theorem]{Criterion}
\newtheorem{definition}[theorem]{Definition}
\newtheorem{example}[theorem]{Example}
\newtheorem{exercise}[theorem]{Exercise}
\newtheorem{lemma}[theorem]{Lemma}
\newtheorem{notation}[theorem]{Notation}
\newtheorem{problem}[theorem]{Problem}
\newtheorem{proposition}[theorem]{Proposition}
\newtheorem{remark}[theorem]{Remark}
\newtheorem{solution}[theorem]{Solution}
\newtheorem{summary}[theorem]{Summary}
\newenvironment{proof}[1][Proof]{\noindent\textbf{#1.} }{\ \rule{0.5em}{0.5em}}


\begin{document}


3-13 使用 http://wiki.chemprime.chemeddl.org/index.php/Table\_of%
\_Standard\_Molar\_Entropies中的数据

自编C++程序计算298.15K标准%
状态下反应熵变输出%
结果如下,熵变的单位%
为 $\unit{J}\cdot \unit{mol}^{-1}\cdot \unit{K}^{-1}.$

2 H2 1 O2 0.5 1 H2O 1

H2+0.5O2==H2O

The entropy change of the process is: -163.35

2 H2 1 Cl2 1 1 HCl 2

H2+Cl2==2HCl

The entropy change of the process is: 20

2 CH4 1 O2 0.5 1 CH3OH 1

CH4+0.5O2==CH3OH

The entropy change of the process is: -162.05.

3.15 将所给气体的状态方%
程与Van der Walls Equation 对比可看%
出,该状态方程是考虑%
气体分子的体积之后%
的修正,但没有考虑内%
压,下面的计算也说明%
了这一点.

$W=\int_{V_{1}}^{V_{2}}PdV=\int_{V_{1}}^{V_{2}}\frac{RT}{V_{m}-a}dV=RT\ln 
\frac{V_{2}-a}{V_{1}-a};$

该体系为双变量系统,%
内能$U$可写为$T,V$的函数,%
由于该过程等温,$dU=\left( \frac{%
\partial U}{\partial V}\right) _{T}dV$

$\left( \frac{\partial U}{\partial V}\right) _{T}$为该气%
体的内压,由热力学基%
本关系式\qquad $dU=TdS-PdV$可求出$%
\left( \frac{\partial U}{\partial V}\right) _{T}=-P+T\left( \frac{\partial S%
}{\partial V}\right) _{T}$

由Maxwell's relation$\implies \left( \frac{\partial U}{\partial V}%
\right) _{T}=-P+T\left( \frac{\partial P}{\partial T}\right) _{V},$代%
入状态方程可得$\left( \frac{%
\partial U}{\partial V}\right) _{T}=-P+T\frac{R}{V-a}=0,$

即内压为0. (remark, 对1mol Van der Walls gas,%
用内压公式可算出$\left( 
\frac{\partial U}{\partial V}\right) _{T}=\frac{a}{V_{m}^{2}},$恰%
好为气体压力修正项.)

故该气体内能仅是温%
度的函数, 等温过程$\Delta
U=0,$by the first law of thermodynamics $\implies Q=W=RT\ln \frac{V_{2}-a}{%
V_{1}-a}.$

该体系的焓可写为$T,P$的%
函数,类似上面的分析,$%
dH=\left( \frac{\partial H}{\partial P}\right) _{T}dP,$

由热力学基本关系式%
\qquad $dH=TdS+VdP$可求出$\left( \frac{\partial H}{%
\partial P}\right) _{T}=V+T\left( \frac{\partial S}{\partial P}\right) _{T}$%
由Maxwell's relation$\implies $

$\left( \frac{\partial H}{\partial P}\right) _{T}=V-T\left( \frac{\partial V%
}{\partial T}\right) _{P}=a,$is a constant$\implies dH=adP.$

由气体状态方程可求%
出初末态压力分别为$%
P_{1}=\frac{RT}{V_{1}-a},P_{2}=\frac{RT}{V_{2}-a}\implies H=a\left(
P_{2}-P_{1}\right) $

$=\frac{aRT\left( V_{2}-V_{1}\right) }{\left( V_{1}-a\right) \left(
V_{2}-a\right) }.$

由\qquad $dU=TdS-PdV\qquad $积分得到熵%
变$\Delta S$满足$0=T\Delta S-W,$或直接%
由Clausius Equality $\Delta S=\frac{Q}{T}$及上面%
推出的$W=Q$得到.故$\Delta S=R\ln \frac{%
V_{2}-a}{V_{1}-a}.$

由$G$的定义$\implies \Delta G=\Delta H-T\Delta S=%
\frac{aRT\left( V_{2}-V_{1}\right) }{\left( V_{1}-a\right) \left(
V_{2}-a\right) }-RT\ln \frac{V_{2}-a}{V_{1}-a}.$

将$T$除到等式左边做积%
分有$\Delta S=\int_{T_{1}}^{T}\frac{dH}{T},$

\bigskip 

\bigskip 

\bigskip 

\bigskip 

\bigskip 

3.26 如果,不妨设在给定%
温度区间内T1和T2是任意%
两个温度值,题意为%
图中I,III两反应反应焓%
变相同,为此设计II IV 
两简单物理过程,且%
假设所有反应均在等%
压条件下进行。由焓%
变的可加性必有II IV两过%
程焓变等值反号,对%
于等压简单变温过程%
如II,IV,有

$\Delta H=\int C_{p}dT$ ,由T1和T2的任意%
性对II,IV,其定压热容%
必等值反号。

再考虑II,IV两过程熵变%
,对于等压简单变温%
过程熵变计算公式为 $%
\Delta S=\int \frac{C_{p}}{T}dT$ 

由于II,IV过程Cp处处等%
值反号,II,IV两过程熵%
变也等值反号,最后%
由熵变的可加性推出I%
,III两过程熵变相同。%
即此反应在给定温度%
区间任意一温度下进%
行熵变为定值。

%\FRAME{ftbpF}

3.40$\left( 1\right) $ $H=H\left( T,P\right) \implies dH=C_{P}dT+\left( 
\frac{\partial H}{\partial P}\right) _{T}dP,$在$V$不变%
的条件下两边同时除%
以$dT,$

$\implies \left( \frac{\partial H}{\partial T}\right) _{V}=C_{P}+\left( 
\frac{\partial H}{\partial P}\right) _{T}\left( \frac{\partial P}{\partial T}%
\right) _{V}$

由3.15的推导有$\left( \frac{\partial H}{%
\partial P}\right) _{T}=V-T\left( \frac{\partial V}{\partial T}\right) _{P},$

\bigskip 代入上式有:$\left( \frac{\partial H}{%
\partial T}\right) _{V}=C_{P}+\left( V-T\left( \frac{\partial V}{\partial T}%
\right) _{P}\right) \left( \frac{\partial P}{\partial T}\right) _{V}\qquad $

再由$H$的定义$H=U+PV\implies \left( \frac{%
\partial H}{\partial T}\right) _{V}=C_{V}+\left( \frac{\partial P}{\partial T%
}\right) _{V}V,\qquad $代入上式中整%
理得

$C_{V}=C_{P}-T\left( \frac{\partial V}{\partial T}\right) _{P}\left( \frac{%
\partial P}{\partial T}\right) _{V}\qquad \left( 3\right) $

即为$C_{p}-C_{V}=T\left( \frac{\partial V}{\partial T}\right)
_{P}\left( \frac{\partial P}{\partial T}\right) _{V}.$

\end{document}
