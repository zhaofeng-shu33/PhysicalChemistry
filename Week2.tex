
\documentclass{ctexart}
%%%%%%%%%%%%%%%%%%%%%%%%%%%%%%%%%%%%%%%%%%%%%%%%%%%%%%%%%%%%%%%%%%%%%%%%%%%%%%%%%%%%%%%%%%%%%%%%%%%%%%%%%%%%%%%%%%%%%%%%%%%%%%%%%%%%%%%%%%%%%%%%%%%%%%%%%%%%%%%%%%%%%%%%%%%%%%%%%%%%%%%%%%%%%%%%%%%%%%%%%%%%%%%%%%%%%%%%%%%%%%%%%%%%%%%%%%%%%%%%%%%%%%%%%%%%
\usepackage{amsmath}
\usepackage{units}
\setcounter{MaxMatrixCols}{10}
%TCIDATA{OutputFilter=LATEX.DLL}
%TCIDATA{Version=5.00.0.2552}
%TCIDATA{<META NAME="SaveForMode" CONTENT="1">}
%TCIDATA{Created=Thursday, September 24, 2015 18:11:16}
%TCIDATA{LastRevised=Thursday, October 15, 2015 22:57:08}
%TCIDATA{<META NAME="GraphicsSave" CONTENT="32">}
%TCIDATA{<META NAME="DocumentShell" CONTENT="Scientific Notebook\Blank Document">}
%TCIDATA{CSTFile=Math with theorems suppressed.cst}
%TCIDATA{PageSetup=72,72,72,72,0}
%TCIDATA{AllPages=
%F=36,\PARA{038<p type="texpara" tag="Body Text" >\hfill \thepage}
%}


\newtheorem{theorem}{Theorem}
\newtheorem{acknowledgement}[theorem]{Acknowledgement}
\newtheorem{algorithm}[theorem]{Algorithm}
\newtheorem{axiom}[theorem]{Axiom}
\newtheorem{case}[theorem]{Case}
\newtheorem{claim}[theorem]{Claim}
\newtheorem{conclusion}[theorem]{Conclusion}
\newtheorem{condition}[theorem]{Condition}
\newtheorem{conjecture}[theorem]{Conjecture}
\newtheorem{corollary}[theorem]{Corollary}
\newtheorem{criterion}[theorem]{Criterion}
\newtheorem{definition}[theorem]{Definition}
\newtheorem{example}[theorem]{Example}
\newtheorem{exercise}[theorem]{Exercise}
\newtheorem{lemma}[theorem]{Lemma}
\newtheorem{notation}[theorem]{Notation}
\newtheorem{problem}[theorem]{Problem}
\newtheorem{proposition}[theorem]{Proposition}
\newtheorem{remark}[theorem]{Remark}
\newtheorem{solution}[theorem]{Solution}
\newtheorem{summary}[theorem]{Summary}
\newenvironment{proof}[1][Proof]{\noindent\textbf{#1.} }{\ \rule{0.5em}{0.5em}}


\begin{document}


\bigskip 物化第二周作业\qquad
\qquad \qquad 赵丰2013012178

Problem 1-10

$\left( 1\right) $对于N$_{2},$由理想气%
体状态方程$\qquad P=\frac{nRT}{V}=\frac{1\unit{%
mol}\times 8.314\unit{J}\cdot \unit{mol}^{-1}\cdot \unit{K}^{-1}\times 273.2%
\unit{K}}{70.3\times 10^{-6}\unit{m}^{3}}=\allowbreak 3.\,\allowbreak
231\,0\times 10^{4}\unit{kPa}$

$\left( 2\right) $若改用Van der Waals' equation, 查%
表知\qquad $a=0.141\unit{Pa}\cdot \unit{m}^{6}\cdot \unit{mol}%
^{-2},b=3.91\times 10^{-5}\unit{m}^{3}\cdot \unit{mol}^{-1}.$

$P=\frac{RT}{V_{m}-b}-\frac{a}{V_{m}^{2}}=\frac{8.314\unit{J}\cdot \unit{mol}%
^{-1}\cdot \unit{K}^{-1}\times 273.2\unit{K}}{70.3\times 10^{-6}\unit{m}%
^{3}\cdot \unit{mol}^{-1}-3.91\times 10^{-5}\unit{m}^{3}\cdot \unit{mol}^{-1}%
}-\frac{0.141\unit{Pa}\cdot \unit{m}^{6}\cdot \unit{mol}^{-2}}{\left(
70.3\times 10^{-6}\unit{m}^{3}\cdot \unit{mol}^{-1}\right) ^{2}}=\allowbreak
4.\,\allowbreak 427\times 10^{4}\unit{kPa}$

$\left( 3\right) $若用压缩因子图%
的方法,在给定的温度$%
T=273.2\unit{K}$和体积$V_{m}=70.3\times 10^{-6}\unit{m}%
^{3}/\unit{mol}$的条件下

查表得$N_{2}$的临界参数%
为$T_{c}=126.1\unit{K},P_{c}=3.39\unit{MPa}.$

压缩因子$Z=\frac{PV_{m}}{RT}=\frac{P_{r}P_{c}V_{m}}{%
RT}=\frac{3.39\unit{MPa}\times 70.3\times 10^{-6}\unit{m}^{3}/\unit{mol}}{%
8.314\unit{J}\cdot \unit{mol}^{-1}\cdot \unit{K}^{-1}\times 273.2\unit{K}}%
P_{r}\approx \allowbreak 0.105P_{r}.$

另一方面$N_{2}$的状态要%
落在$T_{r}=\frac{T}{T_{c}}=\frac{273.2\unit{K}}{126.1\unit{K}}%
\approx 2.\,\allowbreak 17$的等对比温度%
线上$。 $

在$Z-P_{r}$图上画出直线$%
\allowbreak Z=0.105P_{r}$ 求出其与此等%
对比温度线的交点即%
为$N_{2}$的状态,如下图所%
示$:$

%\FRAME


图中画出了直线$\allowbreak
Z=0.105P_{r}$与$T_{r}=2$的等对比温%
度线的交点,对应的对%
比压力大约为$12.5。 $

因此$P=P_{c}P_{r}=3.39\unit{MPa}\times 12.5=\allowbreak
42.\,\allowbreak 375\unit{MPa}\approx 4.24\times 10^{4}\unit{kPa}.$

比较$\left( 1,2,3\right) $中求出的%
结果可知$\left( 3\right) $与实际%
测量值最为接近,用Van der
Waal's equation得到的结果有一%
定偏差$,$而由于外压很%
大$,$用理想气体状态方%
程求出的结果则完全%
偏离真实值$。 $

Problem 2-1 (1) 由热力学第一定%
律:$\Delta U=Q-W=397.5\unit{J}+167.4\unit{J}=Q-83.68\unit{J};$

解得 $Q=\allowbreak 481.\,\allowbreak 22\unit{J}.$即%
放热$\allowbreak 481.\,\allowbreak 22\unit{J}.$

$-\Delta U=Q^{\prime }-125.5\unit{J}$

解得 $Q^{\prime }=-439.\,\allowbreak 4\unit{J},$即吸%
热$439.\,\allowbreak 4\unit{J}.$

Problem 2-5

$W=\int_{V_{1}}^{V_{2}}PdV=\int_{V_{1}}^{V_{2}}\left( \frac{nRT}{V-nb}-\frac{%
n^{2}a}{V^{2}}\right) dV=nRT\ln \left( \frac{V_{2}-b}{V_{1}-b}\right)
+n^{2}a\left( \frac{1}{V_{2}}-\frac{1}{V_{1}}\right) ,$

代入数据 $T=300\unit{K},V_{1}=0.010\unit{m}%
^{3},V_{2}=0.030\unit{m}^{3},$

$a=0.5563\unit{Pa}\cdot \unit{m}^{6}\cdot \unit{mol}^{-2},b=8.4\times 10^{-5}%
\unit{m}^{3}\cdot \unit{mol}^{-1}$

$W=1\unit{mol}\times 8.314\unit{J}\cdot \unit{mol}^{-1}\cdot \unit{K}%
^{-1}\times 300\unit{K}\times \ln \left( \frac{0.030-8.4\times 10^{-5}}{%
0.010-8.4\times 10^{-5}}\right) $

$+\left( 1\unit{mol}\right) ^{2}\times 0.5563\unit{Pa}\cdot \unit{m}%
^{6}\cdot \unit{mol}^{-2}\times \left( \frac{1}{0.030\unit{m}^{3}\cdot \unit{%
mol}^{-1}}-\frac{1}{0.010\unit{m}^{3}\cdot \unit{mol}^{-1}}\right)
=\allowbreak 2717.\,\allowbreak 1\unit{J}$

Problem 2-6

$\left( 1\right) $循环过程系统所%
做的功为曲边三角形ABC%
所围的面积

$\left( 2\right) B\rightarrow C$的$\Delta
U=U_{C}-U_{B}=U_{C}-U_{A}=\Delta U_{A\rightarrow C}=-W\qquad $即%
为曲线AC下的面积的负%
值

$\left( 3\right) \Delta U_{B\rightarrow C}=Q-W^{\prime }\qquad Q=W^{\prime
}-W$,\qquad 即为BC线段下的面%
积与曲线AC下的面积的%
差值.

Added Problem From Atkins' Physical Chemistry:

Calculate the volume occupied by 1.00 mol N$_{2}$ using the van der Waals
equations in the form of a virial

expansion at (a) its critical temperature, (b) its Boyle temperature. Assume
that the pressure is 10atm throughtout. At what temperature is the gas most
perfect? Use the following data:$T_{c}=126.3\unit{K},$

$a=1.408\unit{atm}\cdot \unit{l}^{2}\cdot \unit{mol}^{-2},b=0.0391\unit{l}%
\cdot \unit{mol}^{-1}.$

Solution: $P=\frac{RT}{V_{m}-b}-\frac{a}{V_{m}^{2}}=\frac{RT}{V_{m}}\frac{1}{%
1-\frac{b}{V_{m}}}-\frac{a}{V_{m}^{2}}\overset{Taylor\_Expansion}{=}\frac{RT%
}{V_{m}}\left( 1+\frac{b}{V_{m}}\right) -\frac{a}{V_{m}^{2}}$

=$\frac{RT}{V_{m}}+\frac{RTb-a}{V_{m}^{2}}.\implies PV_{m}=RT+\frac{RTb-a}{%
V_{m}},$ which is in the form of virial expansion.

$\left( a\right) $ at critical temperature $T_{c}=126.3\unit{K},$after
substituting given data into the above equation we can get

a quadratic equation about $V_{m}:$

$10\unit{atm}\times V_{m}^{2}=8.206\times 10^{-2}\unit{l}\cdot \unit{atm}%
\cdot \unit{K}^{-1}\cdot \unit{mol}^{-1}\times 126.3\unit{K}\times
V_{m}+C=\allowbreak 10.\,\allowbreak 364\left( \unit{l}\right) \frac{\unit{%
atm}}{\unit{mol}}\times V_{m}+C,$

where $C=8.206\times 10^{-2}\unit{l}\cdot \unit{atm}\cdot \unit{K}^{-1}\cdot 
\unit{mol}^{-1}\times 126.3\unit{K}\times 0.0391\unit{l}\cdot \unit{mol}%
^{-1}-1.408\unit{atm}\cdot \unit{l}^{2}\cdot \unit{mol}^{-2}$

$=-1.\,\allowbreak 002\,8\unit{l}^{2}\frac{\unit{atm}}{\unit{mol}^{2}}.$

After cancelling out the units, the equation reduces to $%
10V_{m}^{2}-10.364V_{m}+1.0028=0;$

The acceptable root is $V_{m}=0.928\unit{l}\cdot \unit{mol}^{-1}\left( \text{%
another real root is too small and impossible}\right) $

$\left( b\right) $ At Boyle temperature, the first coefficient in the virial
expansion vanishes

$\implies RTb-a=0$

$\implies T=\frac{a}{bR}=\frac{1.408\unit{atm}\cdot \unit{l}^{2}\cdot \unit{%
mol}^{-2}}{0.0391\unit{l}\cdot \unit{mol}^{-1}\times 8.206\times 10^{-2}%
\unit{l}\cdot \unit{atm}\cdot \unit{K}^{-1}\cdot \unit{mol}^{-1}}%
=\allowbreak 438.\,\allowbreak 83\unit{K}.$

Substituting the solved temperature into the approximate equation:$PV_{m}=RT$
gives

$V_{m}=\frac{RT}{P}=\frac{8.206\times 10^{-2}\unit{l}\cdot \unit{atm}\cdot 
\unit{K}^{-1}\cdot \unit{mol}^{-1}\times \allowbreak 438.\,\allowbreak 83%
\unit{K}}{10\unit{atm}}=\allowbreak 3.\,\allowbreak 601\frac{\unit{l}}{\unit{%
mol}},$which is much larger than the results in $\left( a\right) .$

The larger $V_{m}$ in $\left( b\right) $ shows that the same amount of gas
molecules occupy larger volume in the container and thus they have less
intermolecular interaction, which implies that their behavior is much

more like the ideal gas. $\left( \text{the approximate equation }PV_{m}=RT%
\text{ also implies this conclusion.}\right) $

\end{document}
