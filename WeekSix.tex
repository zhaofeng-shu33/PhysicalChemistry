
\documentclass{article}
%%%%%%%%%%%%%%%%%%%%%%%%%%%%%%%%%%%%%%%%%%%%%%%%%%%%%%%%%%%%%%%%%%%%%%%%%%%%%%%%%%%%%%%%%%%%%%%%%%%%%%%%%%%%%%%%%%%%%%%%%%%%%%%%%%%%%%%%%%%%%%%%%%%%%%%%%%%%%%%%%%%%%%%%%%%%%%%%%%%%%%%%%%%%%%%%%%%%%%%%%%%%%%%%%%%%%%%%%%%%%%%%%%%%%%%%%%%%%%%%%%%%%%%%%%%%
\usepackage{amsmath}

\setcounter{MaxMatrixCols}{10}
%TCIDATA{OutputFilter=LATEX.DLL}
%TCIDATA{Version=5.00.0.2552}
%TCIDATA{<META NAME="SaveForMode" CONTENT="1">}
%TCIDATA{Created=Thursday, October 22, 2015 15:58:43}
%TCIDATA{LastRevised=Friday, November 06, 2015 13:26:37}
%TCIDATA{<META NAME="GraphicsSave" CONTENT="32">}
%TCIDATA{<META NAME="DocumentShell" CONTENT="Standard LaTeX\Blank - Standard LaTeX Article">}
%TCIDATA{CSTFile=40 LaTeX article.cst}

\newtheorem{theorem}{Theorem}
\newtheorem{acknowledgement}[theorem]{Acknowledgement}
\newtheorem{algorithm}[theorem]{Algorithm}
\newtheorem{axiom}[theorem]{Axiom}
\newtheorem{case}[theorem]{Case}
\newtheorem{claim}[theorem]{Claim}
\newtheorem{conclusion}[theorem]{Conclusion}
\newtheorem{condition}[theorem]{Condition}
\newtheorem{conjecture}[theorem]{Conjecture}
\newtheorem{corollary}[theorem]{Corollary}
\newtheorem{criterion}[theorem]{Criterion}
\newtheorem{definition}[theorem]{Definition}
\newtheorem{example}[theorem]{Example}
\newtheorem{exercise}[theorem]{Exercise}
\newtheorem{lemma}[theorem]{Lemma}
\newtheorem{notation}[theorem]{Notation}
\newtheorem{problem}[theorem]{Problem}
\newtheorem{proposition}[theorem]{Proposition}
\newtheorem{remark}[theorem]{Remark}
\newtheorem{solution}[theorem]{Solution}
\newtheorem{summary}[theorem]{Summary}
\newenvironment{proof}[1][Proof]{\noindent\textbf{#1.} }{\ \rule{0.5em}{0.5em}}
\input{tcilatex}

\begin{document}


$\bigskip $物化第六周作业\qquad 
赵丰\qquad 2013012178

$\bigskip 3-2\left( a\right) $理想致冷机%
可看作倒开的卡诺机,%
其致冷系数定义为$w=\frac{%
Q_{2}}{A},Q_{2}$为从低温热源吸%
收的热量,$A$为对外输出%
的有用功,若设$Q_{1}$为向%
高温热源释放的热量,%
则$w$又可写为$w=\frac{Q_{2}}{Q_{1}-Q_{2}},$%
对由高低温热源和致%
冷机组成的孤立系,由%
Clausius Inequality$\implies \frac{-Q_{2}}{T_{2}}+\frac{Q_{1}}{T_{1}}\geq 0,$%
其中$T_{1}$为高温热源温%
度而$T_{2}$为低温热源温%
度,从而求出$w\leq \frac{T_{2}}{T_{1}-T_{2}}$%
等号取得当且仅当过%
程可逆.$Q_{2}1\unit{kg}$冰融化所%
需热量,$Q_{2}=6000\times \frac{1000}{18}=\allowbreak 333.3%
\unit{kJ}.$

为极小化$A$,需极大化$%
w\implies A_{\min }=\frac{Q_{2}}{w_{\max }}=\frac{Q_{2}\left(
T_{2}-T_{1}\right) }{T_{2}}=\frac{333.3\unit{kJ}\times \left( 298-273\right) 
}{273}=\allowbreak 30.\,\allowbreak 522\unit{kJ}$

$\left( b\right) Q_{1}=Q_{2}+A=333.3\unit{kJ}+30.52\unit{kJ}=\allowbreak
363.\,\allowbreak 82\unit{kJ}$

以上各热力学量为计%
算方便均取绝对值

由题中所给的致冷机%
的工作环境可知$\eta _{\max }=1-%
\frac{273}{298}=0.084\allowbreak $

$1\unit{kg}$冰融化所需热量$%
Q_{2}=6000\times 18=\allowbreak 108\,\unit{kJ}.$由$\eta =\frac{W}{%
W+Q_{2}}$可解出$W=Q_{2}\left( \frac{1}{1-\eta }-1\right) ,$%
随$\eta $的增大而增大$\implies $

$W$

$3-4$由热力学第二定律Clausius%
等式可知对可逆过程$%
dQ=TdS.$故图中①过程等温%
吸热$Q_{1}$为对应线段下%
矩形的面积,②、④过%
程均为绝热过程无热%
量传递因而也是等熵%
过程,③过程等温放热,$%
Q_{2}$为对应线段下矩形%
的面积的负值.整个循%
环无内能变化,因而体%
系对外输出的功可由%
热力学第一定律算得%
为$Q_{1}+Q_{2}$即为图中封闭曲%
线所围的面积.

3-20即证明不同浓度的溶%
液混合过程的不可逆%
性,首先我们可以设计%
一个化学电池利用浓%
差对外做电功,即利用%
混合热中的一部分对%
外做功,如果不同溶液%
的自动扩散是可逆的,%
那么利用反应放出的%
热量可以让混合后的%
溶液浓度差增大,这样%
的总的效果就是浓差%
电池将热全部转化为%
了功而没有产生其他%
影响,与热力学第二定%
律的Kelvin表述相矛盾.因%
此浓度不同的溶液共%
处时,自动扩散的过程%
是不可逆的.

Atkins 4.3 所需热力学数据有%
水在100$%
%TCIMACRO{}%
%BeginExpansion
{{}^\circ}%
%EndExpansion
$C 时的摩尔汽化热$\Delta
_{l}^{g}H_{m}\left( \text{H}_{2}\text{O}\right) =40.6\unit{kJ}\cdot \unit{mol%
}^{-1},$液态水的近似摩尔%
定压热容$C_{p}=75.291\unit{J}\cdot \unit{K}%
^{-1}\cdot \unit{mol}^{-1}.$设末态温度为%
T,则可列出热平衡方程%
为\qquad

$40.6\unit{kJ}+\left( 373\unit{K}-T\right) \times 75.291\unit{J}\cdot \unit{K%
}^{-1}=2.00\unit{kg}\times 0.385\unit{J}\cdot \unit{K}^{-1}\cdot \unit{g}%
^{-1}\times \left( T-273\unit{K}\right) $,

Solution is: $T=273\unit{K}+56.\,\allowbreak 938\unit{K}=\allowbreak
329.\,\allowbreak 94\unit{K}$

The heat transfered from water to copper is either side of the above
equation, which gives

$\allowbreak 43.\,\allowbreak 842\unit{kJ}.$

$\Delta S\left( \text{H}_{2}\text{O}\right) =-\frac{40.6\unit{kJ}}{373\unit{K%
}}+\int_{373}^{273+56.938}\frac{C_{p}dT}{T}=-108.85\unit{J}\cdot \unit{K}%
^{-1}-75.291\unit{J}\cdot \unit{K}^{-1}\times \log \frac{373}{329.94}%
=\allowbreak -118.\,\allowbreak 09\unit{J}\cdot \unit{K}^{-1}$

$\Delta S\left( \text{Cu}\right) =\int_{273}^{273+56.938}\frac{mC_{p,s}dT}{T}%
=2.00\unit{kg}\times 0.385\unit{J}\cdot \unit{K}^{-1}\cdot \unit{g}%
^{-1}\times \log \frac{329.94}{273}=145.\,87\unit{J}\cdot \unit{K}^{-1}.$

the entropy change of the total system is the addition of the above two,
which gives

$27.78\unit{J}\cdot \unit{K}^{-1}.$

The positive value illustrates this example obeys the second thermodynamic
law.

\bigskip A more accurate solution: Gaseous water is not totally converted to
liquid state. At $T=\allowbreak 329.\,\allowbreak 94K,$the vapour pressure
of water is $15.7520\unit{kPa}$ approximately, and the molar heat capacity
of gaseous water is $\allowbreak 33.\,\allowbreak 5\unit{J}\cdot \unit{K}%
^{-1}\cdot \unit{mol}^{-1}$ approximately. But the water vapour pressure as
a partial pressure depends on the temperature, that is $p\left( T\right) .$%
Assume the gaseous water is ideal gas, by the state law of ideal gas, we
have $101325\unit{Pa}\times V=1\unit{mol}\times R\times 373\unit{K},$and $%
p\left( T\right) \unit{Pa}\times V=n\unit{mol}\times R\times T\unit{K}($%
Dalton's partial pressure law)

where $n$ represents the remaining gaseous water.

And by the condition for thermal equilibrium at final state:

$\left( 1-n\right) \unit{mol}\times 40.6\unit{kJ}\cdot \unit{mol}^{-1}+$

$\left( 1-n\right) \times \left( 373\unit{K}-T\unit{K}\right) \times 75.291%
\unit{J}\cdot \unit{K}^{-1}+n\times \left( 373\unit{K}-T\unit{K}\right)
\times \allowbreak 33.\,\allowbreak 5\unit{J}\cdot \unit{K}^{-1}$

$=2.00\unit{kg}\times 0.385\unit{J}\cdot \unit{K}^{-1}\cdot \unit{g}%
^{-1}\times \left( T-273\right) \unit{K}$

Joining the above two equations together gives $n=\frac{373p}{101325T};$

Substituting $n$ into the third equation gives

$\left( 1-\frac{373p}{101325T}\right) \times 40600+\left( 1-\frac{373p}{%
101325T}\right) \times \left( 373-T\right) \times 75.291+\frac{373p}{101325T}%
\times \left( 373-T\right) \times \allowbreak 33.\,\allowbreak 5$

$=770\left( T-273\right) $ $\left( \ast \right) $

Below we use two methods to get the approximate solution of the above
equation with the help of

the Clapeyron relation between $p$ and $T$: $p=\exp \left( 20.386-\frac{51.32%
}{T}\right) \ast \frac{101325}{760}($units$:\unit{K};p:\unit{Pa})$

First we can plot the figure of both the implicit function $\left( \ast
\right) $ and $p\left( T\right) ,$and read out the x-coordinate \ \qquad\ \ 

of the point of intersection. Below is the result.

\FRAME{ftbpF}{2.5071in}{1.6293in}{0pt}{}{}{Figure}{\special{language
"Scientific Word";type "GRAPHIC";display "USEDEF";valid_file "T";width
2.5071in;height 1.6293in;depth 0pt;original-width 3.7498in;original-height
2.2407in;cropleft "0";croptop "1";cropright "1";cropbottom "0";tempfilename
'NWMN6P01.wmf';tempfile-properties "XPR";}}

Second we can use recursion method to calculate the convergent point.

初始迭代温度为329 .94, 代%
入$p\left( T\right) $ 中求出压强p1%
,再根据p1的值代入热%
平衡方程$\left( \ast \right) $求解%
出温度,再将温度代%
入$p\left( T\right) $如此反复则可%
求出最终的温度.通过%
编程可求出迭代前二%
十步的温度值:

\{329.94, 320.203, 323.964, 322.705, 323.151, 322.996, 323.05,

323.032, 323.038, 323.036, 323.037, 323.036, 323.036, 323.036,

323.036, 323.036, 323.036, 323.036, 323.036, 323.036\}

由此可见收敛相当快,%
且平衡温度近似为323.036K,%
与法一得到的结果十%
分接近.

4.4 选择A和B为系统

From the given condition we can calculate the initial and final states of
both gases in A and B.

\begin{tabular}{lll}
& initial state & final state \\ 
A & $300\unit{K},2\unit{l},2.493\times 10^{6}\unit{Pa}$ & $\allowbreak 900%
\unit{K}.3\unit{l},4.986\times 10^{6}\unit{Pa}$ \\ 
B & $300\unit{K},2\unit{l},2.493\times 10^{5}\unit{Pa}$ & $300\unit{K},1%
\unit{l},4.986\times 10^{6}\unit{Pa}$%
\end{tabular}

We can design a process to calculate the entropy of both gases.

For A, we first fix the pressure and changes the volume from 2 liter to 3
liter,

then $\Delta S_{1}=\int C_{p}\frac{dT}{T}=2\times \left( 20+8.314\right)
\times \ln \frac{3}{2}=\allowbreak 22.\,\allowbreak 961\unit{J}\cdot \unit{K}%
^{-1};$

Then from the intermediate state $\left( T=450\unit{K},V=3\unit{l}%
,P=2.493\times 10^{6}\unit{Pa}\right) ,$we fix the volume next and increases
the pressure to that of final state: $\Delta S_{2}=\int C_{v}\frac{dT}{T}%
=40\times \ln 2=\allowbreak 27.\,\allowbreak 726\unit{J}\cdot \unit{K}^{-1}.$

The total entropy change of A is $\allowbreak \Delta S_{A}=50.687\unit{J}%
\cdot \unit{K}^{-1}.$

Similarly the total entropy change of B is $\Delta S_{B}=-2\times \left(
20+8.314\right) \times \ln 2+40\times \ln 2=\allowbreak -11.\,\allowbreak 526%
\unit{J}\cdot \unit{K}^{-1}$

$\left( b\right) $ $\Delta A_{A}$ is indeterminate from the information
given. 

$\Delta A_{B}=\int -PdV$ $\left( \text{which is also the negative work of
the process,}W_{B}=\int PdV\right) $

$=-nRT\ln \frac{V_{2}}{V_{1}}=2\unit{mol}\times 8.31\unit{J}\cdot \unit{mol}%
^{-1}\cdot \unit{K}^{-1}\times 300\unit{K}\times \ln 2=\allowbreak 3456.0%
\unit{J}.$

$\bigskip \left( c\right) $ $\Delta G_{A}$ is indeterminate from the
information given. 

$\Delta G_{B}=\Delta A_{B}=\allowbreak 3456.0\unit{J}.$

$\left( d\right) \Delta S_{system}=\Delta S_{A}+\Delta S_{B}=39.161\unit{J}%
\cdot \unit{K}^{-1},$

$\bigskip \Delta S_{surrounding}$ $=-\Delta S_{system},$since the whole
process is reversible.

\end{document}
