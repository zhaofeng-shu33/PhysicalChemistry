
\documentclass{article}
%%%%%%%%%%%%%%%%%%%%%%%%%%%%%%%%%%%%%%%%%%%%%%%%%%%%%%%%%%%%%%%%%%%%%%%%%%%%%%%%%%%%%%%%%%%%%%%%%%%%%%%%%%%%%%%%%%%%%%%%%%%%%%%%%%%%%%%%%%%%%%%%%%%%%%%%%%%%%%%%%%%%%%%%%%%%%%%%%%%%%%%%%%%%%%%%%%%%%%%%%%%%%%%%%%%%%%%%%%%%%%%%%%%%%%%%%%%%%%%%%%%%%%%%%%%%
\usepackage{amsmath}

\setcounter{MaxMatrixCols}{10}
%TCIDATA{OutputFilter=LATEX.DLL}
%TCIDATA{Version=5.00.0.2552}
%TCIDATA{<META NAME="SaveForMode" CONTENT="1">}
%TCIDATA{Created=Thursday, October 22, 2015 15:58:43}
%TCIDATA{LastRevised=Friday, November 06, 2015 13:26:37}
%TCIDATA{<META NAME="GraphicsSave" CONTENT="32">}
%TCIDATA{<META NAME="DocumentShell" CONTENT="Standard LaTeX\Blank - Standard LaTeX Article">}
%TCIDATA{CSTFile=40 LaTeX article.cst}

\newtheorem{theorem}{Theorem}
\newtheorem{acknowledgement}[theorem]{Acknowledgement}
\newtheorem{algorithm}[theorem]{Algorithm}
\newtheorem{axiom}[theorem]{Axiom}
\newtheorem{case}[theorem]{Case}
\newtheorem{claim}[theorem]{Claim}
\newtheorem{conclusion}[theorem]{Conclusion}
\newtheorem{condition}[theorem]{Condition}
\newtheorem{conjecture}[theorem]{Conjecture}
\newtheorem{corollary}[theorem]{Corollary}
\newtheorem{criterion}[theorem]{Criterion}
\newtheorem{definition}[theorem]{Definition}
\newtheorem{example}[theorem]{Example}
\newtheorem{exercise}[theorem]{Exercise}
\newtheorem{lemma}[theorem]{Lemma}
\newtheorem{notation}[theorem]{Notation}
\newtheorem{problem}[theorem]{Problem}
\newtheorem{proposition}[theorem]{Proposition}
\newtheorem{remark}[theorem]{Remark}
\newtheorem{solution}[theorem]{Solution}
\newtheorem{summary}[theorem]{Summary}
\newenvironment{proof}[1][Proof]{\noindent\textbf{#1.} }{\ \rule{0.5em}{0.5em}}
\input{tcilatex}

\begin{document}


$\bigskip $\U{7269}\U{5316}\U{7b2c}\U{516d}\U{5468}\U{4f5c}\U{4e1a}\qquad 
\U{8d75}\U{4e30}\qquad 2013012178

$\bigskip 3-2\left( a\right) $\U{7406}\U{60f3}\U{81f4}\U{51b7}\U{673a}%
\U{53ef}\U{770b}\U{4f5c}\U{5012}\U{5f00}\U{7684}\U{5361}\U{8bfa}\U{673a},%
\U{5176}\U{81f4}\U{51b7}\U{7cfb}\U{6570}\U{5b9a}\U{4e49}\U{4e3a}$w=\frac{%
Q_{2}}{A},Q_{2}$\U{4e3a}\U{4ece}\U{4f4e}\U{6e29}\U{70ed}\U{6e90}\U{5438}%
\U{6536}\U{7684}\U{70ed}\U{91cf},$A$\U{4e3a}\U{5bf9}\U{5916}\U{8f93}\U{51fa}%
\U{7684}\U{6709}\U{7528}\U{529f},\U{82e5}\U{8bbe}$Q_{1}$\U{4e3a}\U{5411}%
\U{9ad8}\U{6e29}\U{70ed}\U{6e90}\U{91ca}\U{653e}\U{7684}\U{70ed}\U{91cf},%
\U{5219}$w$\U{53c8}\U{53ef}\U{5199}\U{4e3a}$w=\frac{Q_{2}}{Q_{1}-Q_{2}},$%
\U{5bf9}\U{7531}\U{9ad8}\U{4f4e}\U{6e29}\U{70ed}\U{6e90}\U{548c}\U{81f4}%
\U{51b7}\U{673a}\U{7ec4}\U{6210}\U{7684}\U{5b64}\U{7acb}\U{7cfb},\U{7531}%
Clausius Inequality$\implies \frac{-Q_{2}}{T_{2}}+\frac{Q_{1}}{T_{1}}\geq 0,$%
\U{5176}\U{4e2d}$T_{1}$\U{4e3a}\U{9ad8}\U{6e29}\U{70ed}\U{6e90}\U{6e29}%
\U{5ea6}\U{800c}$T_{2}$\U{4e3a}\U{4f4e}\U{6e29}\U{70ed}\U{6e90}\U{6e29}%
\U{5ea6},\U{4ece}\U{800c}\U{6c42}\U{51fa}$w\leq \frac{T_{2}}{T_{1}-T_{2}}$%
\U{7b49}\U{53f7}\U{53d6}\U{5f97}\U{5f53}\U{4e14}\U{4ec5}\U{5f53}\U{8fc7}%
\U{7a0b}\U{53ef}\U{9006}.$Q_{2}1\unit{kg}$\U{51b0}\U{878d}\U{5316}\U{6240}%
\U{9700}\U{70ed}\U{91cf},$Q_{2}=6000\times \frac{1000}{18}=\allowbreak 333.3%
\unit{kJ}.$

\U{4e3a}\U{6781}\U{5c0f}\U{5316}$A$,\U{9700}\U{6781}\U{5927}\U{5316}$%
w\implies A_{\min }=\frac{Q_{2}}{w_{\max }}=\frac{Q_{2}\left(
T_{2}-T_{1}\right) }{T_{2}}=\frac{333.3\unit{kJ}\times \left( 298-273\right) 
}{273}=\allowbreak 30.\,\allowbreak 522\unit{kJ}$

$\left( b\right) Q_{1}=Q_{2}+A=333.3\unit{kJ}+30.52\unit{kJ}=\allowbreak
363.\,\allowbreak 82\unit{kJ}$

\U{4ee5}\U{4e0a}\U{5404}\U{70ed}\U{529b}\U{5b66}\U{91cf}\U{4e3a}\U{8ba1}%
\U{7b97}\U{65b9}\U{4fbf}\U{5747}\U{53d6}\U{7edd}\U{5bf9}\U{503c}

\U{7531}\U{9898}\U{4e2d}\U{6240}\U{7ed9}\U{7684}\U{81f4}\U{51b7}\U{673a}%
\U{7684}\U{5de5}\U{4f5c}\U{73af}\U{5883}\U{53ef}\U{77e5}$\eta _{\max }=1-%
\frac{273}{298}=0.084\allowbreak $

$1\unit{kg}$\U{51b0}\U{878d}\U{5316}\U{6240}\U{9700}\U{70ed}\U{91cf}$%
Q_{2}=6000\times 18=\allowbreak 108\,\unit{kJ}.$\U{7531}$\eta =\frac{W}{%
W+Q_{2}}$\U{53ef}\U{89e3}\U{51fa}$W=Q_{2}\left( \frac{1}{1-\eta }-1\right) ,$%
\U{968f}$\eta $\U{7684}\U{589e}\U{5927}\U{800c}\U{589e}\U{5927}$\implies $

$W$

$3-4$\U{7531}\U{70ed}\U{529b}\U{5b66}\U{7b2c}\U{4e8c}\U{5b9a}\U{5f8b}Clausius%
\U{7b49}\U{5f0f}\U{53ef}\U{77e5}\U{5bf9}\U{53ef}\U{9006}\U{8fc7}\U{7a0b}$%
dQ=TdS.$\U{6545}\U{56fe}\U{4e2d}\U{2460}\U{8fc7}\U{7a0b}\U{7b49}\U{6e29}%
\U{5438}\U{70ed}$Q_{1}$\U{4e3a}\U{5bf9}\U{5e94}\U{7ebf}\U{6bb5}\U{4e0b}%
\U{77e9}\U{5f62}\U{7684}\U{9762}\U{79ef},\U{2461}\U{3001}\U{2463}\U{8fc7}%
\U{7a0b}\U{5747}\U{4e3a}\U{7edd}\U{70ed}\U{8fc7}\U{7a0b}\U{65e0}\U{70ed}%
\U{91cf}\U{4f20}\U{9012}\U{56e0}\U{800c}\U{4e5f}\U{662f}\U{7b49}\U{71b5}%
\U{8fc7}\U{7a0b},\U{2462}\U{8fc7}\U{7a0b}\U{7b49}\U{6e29}\U{653e}\U{70ed},$%
Q_{2}$\U{4e3a}\U{5bf9}\U{5e94}\U{7ebf}\U{6bb5}\U{4e0b}\U{77e9}\U{5f62}%
\U{7684}\U{9762}\U{79ef}\U{7684}\U{8d1f}\U{503c}.\U{6574}\U{4e2a}\U{5faa}%
\U{73af}\U{65e0}\U{5185}\U{80fd}\U{53d8}\U{5316},\U{56e0}\U{800c}\U{4f53}%
\U{7cfb}\U{5bf9}\U{5916}\U{8f93}\U{51fa}\U{7684}\U{529f}\U{53ef}\U{7531}%
\U{70ed}\U{529b}\U{5b66}\U{7b2c}\U{4e00}\U{5b9a}\U{5f8b}\U{7b97}\U{5f97}%
\U{4e3a}$Q_{1}+Q_{2}$\U{5373}\U{4e3a}\U{56fe}\U{4e2d}\U{5c01}\U{95ed}\U{66f2}%
\U{7ebf}\U{6240}\U{56f4}\U{7684}\U{9762}\U{79ef}.

3-20\U{5373}\U{8bc1}\U{660e}\U{4e0d}\U{540c}\U{6d53}\U{5ea6}\U{7684}\U{6eb6}%
\U{6db2}\U{6df7}\U{5408}\U{8fc7}\U{7a0b}\U{7684}\U{4e0d}\U{53ef}\U{9006}%
\U{6027},\U{9996}\U{5148}\U{6211}\U{4eec}\U{53ef}\U{4ee5}\U{8bbe}\U{8ba1}%
\U{4e00}\U{4e2a}\U{5316}\U{5b66}\U{7535}\U{6c60}\U{5229}\U{7528}\U{6d53}%
\U{5dee}\U{5bf9}\U{5916}\U{505a}\U{7535}\U{529f},\U{5373}\U{5229}\U{7528}%
\U{6df7}\U{5408}\U{70ed}\U{4e2d}\U{7684}\U{4e00}\U{90e8}\U{5206}\U{5bf9}%
\U{5916}\U{505a}\U{529f},\U{5982}\U{679c}\U{4e0d}\U{540c}\U{6eb6}\U{6db2}%
\U{7684}\U{81ea}\U{52a8}\U{6269}\U{6563}\U{662f}\U{53ef}\U{9006}\U{7684},%
\U{90a3}\U{4e48}\U{5229}\U{7528}\U{53cd}\U{5e94}\U{653e}\U{51fa}\U{7684}%
\U{70ed}\U{91cf}\U{53ef}\U{4ee5}\U{8ba9}\U{6df7}\U{5408}\U{540e}\U{7684}%
\U{6eb6}\U{6db2}\U{6d53}\U{5ea6}\U{5dee}\U{589e}\U{5927},\U{8fd9}\U{6837}%
\U{7684}\U{603b}\U{7684}\U{6548}\U{679c}\U{5c31}\U{662f}\U{6d53}\U{5dee}%
\U{7535}\U{6c60}\U{5c06}\U{70ed}\U{5168}\U{90e8}\U{8f6c}\U{5316}\U{4e3a}%
\U{4e86}\U{529f}\U{800c}\U{6ca1}\U{6709}\U{4ea7}\U{751f}\U{5176}\U{4ed6}%
\U{5f71}\U{54cd},\U{4e0e}\U{70ed}\U{529b}\U{5b66}\U{7b2c}\U{4e8c}\U{5b9a}%
\U{5f8b}\U{7684}Kelvin\U{8868}\U{8ff0}\U{76f8}\U{77db}\U{76fe}.\U{56e0}%
\U{6b64}\U{6d53}\U{5ea6}\U{4e0d}\U{540c}\U{7684}\U{6eb6}\U{6db2}\U{5171}%
\U{5904}\U{65f6},\U{81ea}\U{52a8}\U{6269}\U{6563}\U{7684}\U{8fc7}\U{7a0b}%
\U{662f}\U{4e0d}\U{53ef}\U{9006}\U{7684}.

Atkins 4.3 \U{6240}\U{9700}\U{70ed}\U{529b}\U{5b66}\U{6570}\U{636e}\U{6709}%
\U{6c34}\U{5728}100$%
%TCIMACRO{}%
%BeginExpansion
{{}^\circ}%
%EndExpansion
$C \U{65f6}\U{7684}\U{6469}\U{5c14}\U{6c7d}\U{5316}\U{70ed}$\Delta
_{l}^{g}H_{m}\left( \text{H}_{2}\text{O}\right) =40.6\unit{kJ}\cdot \unit{mol%
}^{-1},$\U{6db2}\U{6001}\U{6c34}\U{7684}\U{8fd1}\U{4f3c}\U{6469}\U{5c14}%
\U{5b9a}\U{538b}\U{70ed}\U{5bb9}$C_{p}=75.291\unit{J}\cdot \unit{K}%
^{-1}\cdot \unit{mol}^{-1}.$\U{8bbe}\U{672b}\U{6001}\U{6e29}\U{5ea6}\U{4e3a}%
T,\U{5219}\U{53ef}\U{5217}\U{51fa}\U{70ed}\U{5e73}\U{8861}\U{65b9}\U{7a0b}%
\U{4e3a}\qquad

$40.6\unit{kJ}+\left( 373\unit{K}-T\right) \times 75.291\unit{J}\cdot \unit{K%
}^{-1}=2.00\unit{kg}\times 0.385\unit{J}\cdot \unit{K}^{-1}\cdot \unit{g}%
^{-1}\times \left( T-273\unit{K}\right) $,

Solution is: $T=273\unit{K}+56.\,\allowbreak 938\unit{K}=\allowbreak
329.\,\allowbreak 94\unit{K}$

The heat transfered from water to copper is either side of the above
equation, which gives

$\allowbreak 43.\,\allowbreak 842\unit{kJ}.$

$\Delta S\left( \text{H}_{2}\text{O}\right) =-\frac{40.6\unit{kJ}}{373\unit{K%
}}+\int_{373}^{273+56.938}\frac{C_{p}dT}{T}=-108.85\unit{J}\cdot \unit{K}%
^{-1}-75.291\unit{J}\cdot \unit{K}^{-1}\times \log \frac{373}{329.94}%
=\allowbreak -118.\,\allowbreak 09\unit{J}\cdot \unit{K}^{-1}$

$\Delta S\left( \text{Cu}\right) =\int_{273}^{273+56.938}\frac{mC_{p,s}dT}{T}%
=2.00\unit{kg}\times 0.385\unit{J}\cdot \unit{K}^{-1}\cdot \unit{g}%
^{-1}\times \log \frac{329.94}{273}=145.\,87\unit{J}\cdot \unit{K}^{-1}.$

the entropy change of the total system is the addition of the above two,
which gives

$27.78\unit{J}\cdot \unit{K}^{-1}.$

The positive value illustrates this example obeys the second thermodynamic
law.

\bigskip A more accurate solution: Gaseous water is not totally converted to
liquid state. At $T=\allowbreak 329.\,\allowbreak 94K,$the vapour pressure
of water is $15.7520\unit{kPa}$ approximately, and the molar heat capacity
of gaseous water is $\allowbreak 33.\,\allowbreak 5\unit{J}\cdot \unit{K}%
^{-1}\cdot \unit{mol}^{-1}$ approximately. But the water vapour pressure as
a partial pressure depends on the temperature, that is $p\left( T\right) .$%
Assume the gaseous water is ideal gas, by the state law of ideal gas, we
have $101325\unit{Pa}\times V=1\unit{mol}\times R\times 373\unit{K},$and $%
p\left( T\right) \unit{Pa}\times V=n\unit{mol}\times R\times T\unit{K}($%
Dalton's partial pressure law)

where $n$ represents the remaining gaseous water.

And by the condition for thermal equilibrium at final state:

$\left( 1-n\right) \unit{mol}\times 40.6\unit{kJ}\cdot \unit{mol}^{-1}+$

$\left( 1-n\right) \times \left( 373\unit{K}-T\unit{K}\right) \times 75.291%
\unit{J}\cdot \unit{K}^{-1}+n\times \left( 373\unit{K}-T\unit{K}\right)
\times \allowbreak 33.\,\allowbreak 5\unit{J}\cdot \unit{K}^{-1}$

$=2.00\unit{kg}\times 0.385\unit{J}\cdot \unit{K}^{-1}\cdot \unit{g}%
^{-1}\times \left( T-273\right) \unit{K}$

Joining the above two equations together gives $n=\frac{373p}{101325T};$

Substituting $n$ into the third equation gives

$\left( 1-\frac{373p}{101325T}\right) \times 40600+\left( 1-\frac{373p}{%
101325T}\right) \times \left( 373-T\right) \times 75.291+\frac{373p}{101325T}%
\times \left( 373-T\right) \times \allowbreak 33.\,\allowbreak 5$

$=770\left( T-273\right) $ $\left( \ast \right) $

Below we use two methods to get the approximate solution of the above
equation with the help of

the Clapeyron relation between $p$ and $T$: $p=\exp \left( 20.386-\frac{51.32%
}{T}\right) \ast \frac{101325}{760}($units$:\unit{K};p:\unit{Pa})$

First we can plot the figure of both the implicit function $\left( \ast
\right) $ and $p\left( T\right) ,$and read out the x-coordinate \ \qquad\ \ 

of the point of intersection. Below is the result.

\FRAME{ftbpF}{2.5071in}{1.6293in}{0pt}{}{}{Figure}{\special{language
"Scientific Word";type "GRAPHIC";display "USEDEF";valid_file "T";width
2.5071in;height 1.6293in;depth 0pt;original-width 3.7498in;original-height
2.2407in;cropleft "0";croptop "1";cropright "1";cropbottom "0";tempfilename
'NWMN6P01.wmf';tempfile-properties "XPR";}}

Second we can use recursion method to calculate the convergent point.

\U{521d}\U{59cb}\U{8fed}\U{4ee3}\U{6e29}\U{5ea6}\U{4e3a}329 .94, \U{4ee3}%
\U{5165}$p\left( T\right) $ \U{4e2d}\U{6c42}\U{51fa}\U{538b}\U{5f3a}p1%
\U{ff0c}\U{518d}\U{6839}\U{636e}p1\U{7684}\U{503c}\U{4ee3}\U{5165}\U{70ed}%
\U{5e73}\U{8861}\U{65b9}\U{7a0b}$\left( \ast \right) $\U{6c42}\U{89e3}%
\U{51fa}\U{6e29}\U{5ea6}\U{ff0c}\U{518d}\U{5c06}\U{6e29}\U{5ea6}\U{4ee3}%
\U{5165}$p\left( T\right) $\U{5982}\U{6b64}\U{53cd}\U{590d}\U{5219}\U{53ef}%
\U{6c42}\U{51fa}\U{6700}\U{7ec8}\U{7684}\U{6e29}\U{5ea6}.\U{901a}\U{8fc7}%
\U{7f16}\U{7a0b}\U{53ef}\U{6c42}\U{51fa}\U{8fed}\U{4ee3}\U{524d}\U{4e8c}%
\U{5341}\U{6b65}\U{7684}\U{6e29}\U{5ea6}\U{503c}:

\{329.94, 320.203, 323.964, 322.705, 323.151, 322.996, 323.05,

323.032, 323.038, 323.036, 323.037, 323.036, 323.036, 323.036,

323.036, 323.036, 323.036, 323.036, 323.036, 323.036\}

\U{7531}\U{6b64}\U{53ef}\U{89c1}\U{6536}\U{655b}\U{76f8}\U{5f53}\U{5feb},%
\U{4e14}\U{5e73}\U{8861}\U{6e29}\U{5ea6}\U{8fd1}\U{4f3c}\U{4e3a}323.036K,%
\U{4e0e}\U{6cd5}\U{4e00}\U{5f97}\U{5230}\U{7684}\U{7ed3}\U{679c}\U{5341}%
\U{5206}\U{63a5}\U{8fd1}.

4.4 \U{9009}\U{62e9}A\U{548c}B\U{4e3a}\U{7cfb}\U{7edf}

From the given condition we can calculate the initial and final states of
both gases in A and B.

\begin{tabular}{lll}
& initial state & final state \\ 
A & $300\unit{K},2\unit{l},2.493\times 10^{6}\unit{Pa}$ & $\allowbreak 900%
\unit{K}.3\unit{l},4.986\times 10^{6}\unit{Pa}$ \\ 
B & $300\unit{K},2\unit{l},2.493\times 10^{5}\unit{Pa}$ & $300\unit{K},1%
\unit{l},4.986\times 10^{6}\unit{Pa}$%
\end{tabular}

We can design a process to calculate the entropy of both gases.

For A, we first fix the pressure and changes the volume from 2 liter to 3
liter,

then $\Delta S_{1}=\int C_{p}\frac{dT}{T}=2\times \left( 20+8.314\right)
\times \ln \frac{3}{2}=\allowbreak 22.\,\allowbreak 961\unit{J}\cdot \unit{K}%
^{-1};$

Then from the intermediate state $\left( T=450\unit{K},V=3\unit{l}%
,P=2.493\times 10^{6}\unit{Pa}\right) ,$we fix the volume next and increases
the pressure to that of final state: $\Delta S_{2}=\int C_{v}\frac{dT}{T}%
=40\times \ln 2=\allowbreak 27.\,\allowbreak 726\unit{J}\cdot \unit{K}^{-1}.$

The total entropy change of A is $\allowbreak \Delta S_{A}=50.687\unit{J}%
\cdot \unit{K}^{-1}.$

Similarly the total entropy change of B is $\Delta S_{B}=-2\times \left(
20+8.314\right) \times \ln 2+40\times \ln 2=\allowbreak -11.\,\allowbreak 526%
\unit{J}\cdot \unit{K}^{-1}$

$\left( b\right) $ $\Delta A_{A}$ is indeterminate from the information
given. 

$\Delta A_{B}=\int -PdV$ $\left( \text{which is also the negative work of
the process,}W_{B}=\int PdV\right) $

$=-nRT\ln \frac{V_{2}}{V_{1}}=2\unit{mol}\times 8.31\unit{J}\cdot \unit{mol}%
^{-1}\cdot \unit{K}^{-1}\times 300\unit{K}\times \ln 2=\allowbreak 3456.0%
\unit{J}.$

$\bigskip \left( c\right) $ $\Delta G_{A}$ is indeterminate from the
information given. 

$\Delta G_{B}=\Delta A_{B}=\allowbreak 3456.0\unit{J}.$

$\left( d\right) \Delta S_{system}=\Delta S_{A}+\Delta S_{B}=39.161\unit{J}%
\cdot \unit{K}^{-1},$

$\bigskip \Delta S_{surrounding}$ $=-\Delta S_{system},$since the whole
process is reversible.

\end{document}
