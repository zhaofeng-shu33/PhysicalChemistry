
\documentclass{article}
%%%%%%%%%%%%%%%%%%%%%%%%%%%%%%%%%%%%%%%%%%%%%%%%%%%%%%%%%%%%%%%%%%%%%%%%%%%%%%%%%%%%%%%%%%%%%%%%%%%%%%%%%%%%%%%%%%%%%%%%%%%%%%%%%%%%%%%%%%%%%%%%%%%%%%%%%%%%%%%%%%%%%%%%%%%%%%%%%%%%%%%%%%%%%%%%%%%%%%%%%%%%%%%%%%%%%%%%%%%%%%%%%%%%%%%%%%%%%%%%%%%%%%%%%%%%
\usepackage{amsmath}

\setcounter{MaxMatrixCols}{10}
%TCIDATA{OutputFilter=LATEX.DLL}
%TCIDATA{Version=5.00.0.2552}
%TCIDATA{<META NAME="SaveForMode" CONTENT="1">}
%TCIDATA{Created=Saturday, October 17, 2015 09:30:50}
%TCIDATA{LastRevised=Sunday, November 22, 2015 22:19:44}
%TCIDATA{<META NAME="GraphicsSave" CONTENT="32">}
%TCIDATA{<META NAME="DocumentShell" CONTENT="Scientific Notebook\Blank Document">}
%TCIDATA{CSTFile=Math with theorems suppressed.cst}
%TCIDATA{PageSetup=72,72,72,72,0}
%TCIDATA{AllPages=
%F=36,\PARA{038<p type="texpara" tag="Body Text" >\hfill \thepage}
%}


\newtheorem{theorem}{Theorem}
\newtheorem{acknowledgement}[theorem]{Acknowledgement}
\newtheorem{algorithm}[theorem]{Algorithm}
\newtheorem{axiom}[theorem]{Axiom}
\newtheorem{case}[theorem]{Case}
\newtheorem{claim}[theorem]{Claim}
\newtheorem{conclusion}[theorem]{Conclusion}
\newtheorem{condition}[theorem]{Condition}
\newtheorem{conjecture}[theorem]{Conjecture}
\newtheorem{corollary}[theorem]{Corollary}
\newtheorem{criterion}[theorem]{Criterion}
\newtheorem{definition}[theorem]{Definition}
\newtheorem{example}[theorem]{Example}
\newtheorem{exercise}[theorem]{Exercise}
\newtheorem{lemma}[theorem]{Lemma}
\newtheorem{notation}[theorem]{Notation}
\newtheorem{problem}[theorem]{Problem}
\newtheorem{proposition}[theorem]{Proposition}
\newtheorem{remark}[theorem]{Remark}
\newtheorem{solution}[theorem]{Solution}
\newtheorem{summary}[theorem]{Summary}
\newenvironment{proof}[1][Proof]{\noindent\textbf{#1.} }{\ \rule{0.5em}{0.5em}}
\input{tcilatex}

\begin{document}


\bigskip 物化第五周作业\qquad 
赵丰\qquad 2013012178

2-24 $\Delta H=Q_{p}-W^{\prime }=-10\unit{J}-20\unit{J}=-30\unit{J}$

Atkins: 2-22

$\left( a\right) \Delta _{r}H_{m}^{\ominus }=\Delta _{f}H_{m}^{\ominus
}\left( \text{SiH}_{3}\text{OH},g\right) -\Delta _{f}H_{m}^{\ominus }\left( 
\text{SiH}_{4},g\right) =-282\unit{kJ}\cdot \unit{mol}^{-1}-34.3\unit{kJ}%
\cdot \unit{mol}^{-1}=\allowbreak -316.\,\allowbreak 3\unit{kJ}\cdot \unit{%
mol}^{-1}$

$\left( b\right) \Delta _{r}H_{m}^{\ominus }=\Delta _{f}H_{m}^{\ominus
}\left( \text{SiH}_{2}\text{O},g\right) +\Delta _{f}H_{m}^{\ominus }\left( 
\text{H}_{2}\text{O},l\right) -\Delta _{f}H_{m}^{\ominus }\left( \text{SiH}%
_{4},g\right) $

$=-98.3\unit{kJ}\cdot \unit{mol}^{-1}-285.84\unit{kJ}\cdot \unit{mol}%
^{-1}-34.3\unit{kJ}\cdot \unit{mol}^{-1}=\allowbreak -418.\,\allowbreak 44%
\unit{kJ}\cdot \unit{mol}^{-1}$

$\left( c\right) \Delta _{r}H_{m}^{\ominus }=\Delta _{f}H_{m}^{\ominus
}\left( \text{SiH}_{2}\text{O},g\right) -\Delta _{f}H_{m}^{\ominus }\left( 
\text{SiH}_{3}\text{OH},g\right) =-98.3\unit{kJ}\cdot \unit{mol}^{-1}+282%
\unit{kJ}\cdot \unit{mol}^{-1}=\allowbreak 183.\,\allowbreak 7\unit{kJ}\cdot 
\unit{mol}^{-1}$

Textbook 2-10 $3C\left( \text{石墨}\right) $ + $\frac{3}{2}%
H_{2}\left( g\right) $ +$\frac{1}{2}N_{2}\left( g\right) \qquad \rightarrow $
$CH_{2}=CHCN\left( l\right) \qquad $

$\Delta _{f}H_{m}\left( CH_{2}=CHCN,l\right) =3\Delta _{c}H_{m}\left( \text{%
C(石墨)}\right) +\frac{3}{2}\Delta _{c}H_{m}\left( \text{H}%
_{2},g\right) -\Delta _{c}H_{m}\left( CH_{2}=CHCN,l\right) $

$=3\times \left( -393.5\right) \unit{kJ}\cdot \unit{mol}^{-1}+\frac{3}{2}%
\times \left( -285.8\unit{kJ}\cdot \unit{mol}^{-1}\right) -\left( -1761\unit{%
kJ}\cdot \unit{mol}^{-1}\right) =\allowbreak 151.\,\allowbreak 8\unit{kJ}%
\cdot \unit{mol}^{-1}$

\bigskip 由盖斯定律可求出%
\qquad $\Delta _{f}H_{m}\left( CH_{2}=CHCN,g\right) =\Delta _{f}H_{m}\left(
CH_{2}=CHCN,l\right) +\Delta _{l}^{g}H_{m}\left( CH_{2}=CHCN,298.2\unit{K}%
\right) $

=$151.\,\allowbreak 8\unit{kJ}\cdot \unit{mol}^{-1}+32.8\unit{kJ}\cdot \unit{%
mol}^{-1}=\allowbreak 184.\,\allowbreak 6\unit{kJ}\cdot \unit{mol}^{-1}$

对于给定的反应\qquad $\Delta
_{r}H_{m}^{\ominus }=\Delta _{f}H_{m}^{\ominus }\left( CH_{2}=CHCN,g\right)
-\Delta _{f}H_{m}^{\ominus }\left( C_{2}H_{2},g\right) -\Delta
_{f}H_{m}^{\ominus }\left( HCN,g\right) $

=$184.6\unit{kJ}\cdot \unit{mol}^{-1}-226.8\unit{kJ}\cdot \unit{mol}%
^{-1}-129.7\unit{kJ}\cdot \unit{mol}^{-1}=\allowbreak -171.\,\allowbreak 9%
\unit{kJ}\cdot \unit{mol}^{-1}.$

2-17 水的物质的量为\qquad $\frac{%
58.5\unit{g}\times 4}{18\unit{g}/\unit{mol}}=\allowbreak 13.0\unit{mol}$

设该溶液的比热容为$%
\gamma ,可 $设计过程如下图%
求$\gamma ,$

$\bigskip $\FRAME{ftbpF}{3.1324in}{1.5532in}{0pt}{}{}{Figure}{\special%
{language "Scientific Word";type "GRAPHIC";display "USEDEF";valid_file
"T";width 3.1324in;height 1.5532in;depth 0pt;original-width
3.333in;original-height 2.5417in;cropleft "0";croptop "1";cropright
"1";cropbottom "0";tempfilename 'NWCQI103.wmf';tempfile-properties "XPR";}}

$\Delta H=-\Delta H_{1}+\Delta H_{2}+\Delta H_{3}$

\bigskip =$-\Delta _{sol}H\left( 20^{%
%TCIMACRO{}%
%BeginExpansion
{{}^\circ}%
%EndExpansion
}C\right) +n\left( NaCl\right) C_{p,m}\left( NaCl\right) \Delta T+n\left(
H_{2}0\right) C_{p,m}\left( H_{2}O\right) \Delta T+\Delta _{sol}H\left( 25^{%
%TCIMACRO{}%
%BeginExpansion
{{}^\circ}%
%EndExpansion
}C\right) $

=$-3230\unit{J}+1\unit{mol}\times 5\unit{K}\times 50.0\unit{J}\cdot \unit{K}%
^{-1}\cdot \unit{mol}^{-1}+13.0\unit{mol}\times 5\unit{K}\times 75.5\unit{J}%
\cdot \unit{K}^{-1}\cdot \unit{mol}^{-1}+2920\unit{J}=\allowbreak
4847.\,\allowbreak 5\unit{J}$

$\implies \gamma =\frac{4847.\,\allowbreak 5\unit{J}}{5\unit{K}\times \frac{%
58.5\unit{g}}{0.2}}=\allowbreak 3.\,\allowbreak 314\,5\unit{J}\cdot \unit{K}%
^{-1}\cdot \unit{g}^{-1}.$

2-28 由理想气体状态方程%
$\qquad \frac{P}{V}=c\implies c=\frac{P^{2}}{PV}=\frac{P_{1}^{2}}{RT_{1}}.$

$W=\int_{V_{1}}^{V_{2}}PdV=\int_{V_{1}}^{V_{2}}cVdV=\frac{c}{2}\left(
V_{2}^{2}-V_{1}^{2}\right) =\frac{P_{1}^{2}}{2RT_{1}}\left( \frac{%
R^{2}T_{2}^{2}}{P_{2}^{2}}-\frac{R^{2}T_{1}^{2}}{P_{1}^{2}}\right) =\frac{%
RT_{1}}{2}\left( \left( \frac{P_{1}T_{2}}{P_{2}T_{1}}\right) ^{2}-1\right) $

$\Delta U=\frac{3}{2}R\Delta T=\frac{3}{2}R\left( T_{2}-T_{1}\right) .$

因\qquad $\frac{P_{1}^{2}}{RT_{1}}=\frac{P_{2}^{2}}{RT_{2}}\implies 
\frac{P_{1}^{2}}{P_{2}^{2}}=\frac{T_{1}}{T_{2}},$代入$W$中%
得\qquad $W=\frac{RT_{1}}{2}\left( \frac{T_{2}}{T_{1}}-1\right) $

By first thermodynamic law and the definition of heat capacity $C=\frac{%
\Delta Q}{\Delta T}=\frac{W+\Delta U}{\Delta T}$

=$\frac{\frac{RT_{1}}{2}\left( \frac{T_{2}}{T_{1}}-1\right) +\frac{3}{2}%
R\left( T_{2}-T_{1}\right) }{T_{2}-T_{1}}=\allowbreak 2R.$

2-34 $\left( 1\right) \qquad \Delta _{sol}H_{1}=3862\unit{J}+1992\unit{J}%
-3012\unit{J}+1019\unit{J}=\allowbreak 3861\unit{J}.$

$\left( 2\right) $若将$1$ mol NaCl 溶于10kg水%
中,$\Delta H_{2}=10\Delta _{sol}H_{2},$其中$\Delta
_{sol}H_{2}$表示0.1mol NaCl 溶于1kg 水中%
的溶解热

$\Delta H_{2}/\unit{J}=10\times \left( 386.2+1992\times
0.1^{3/2}-30.12+1019\times 0.1^{5/2}\right) =\allowbreak 4222.\,\allowbreak
9.$

故稀释热为\qquad $\Delta _{dil}H=\Delta
_{sol}H_{2}-\Delta _{sol}H_{1}=\allowbreak \allowbreak 4222.\,\allowbreak 9%
\unit{J}-\allowbreak 3861\unit{J}=\allowbreak 361.\,\allowbreak 9\unit{J}.$

$\left( 3\right) \frac{\partial \Delta _{sol}H}{\partial n}_{|n=1}=3862\unit{%
J}+1992\times \frac{3}{2}\unit{J}-3012\times 2\unit{J}+1019\times \frac{5}{2}%
\unit{J}=\allowbreak 3373.\,\allowbreak 5\unit{J}$

$\left( 4\right) $若将$1$ mol NaCl 溶于n kg水%
中 $\Delta H_{3}/\unit{J}=n\Delta _{sol}H\left( \frac{1}{n}\right) ,$%
因n很大,相当于取极限$%
n->\infty ,$

$\underset{n->\infty }{\lim }\Delta H_{3}/\unit{J}=3862.$

\end{document}
