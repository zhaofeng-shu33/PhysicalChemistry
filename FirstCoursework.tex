
\documentclass{ctexart}
%%%%%%%%%%%%%%%%%%%%%%%%%%%%%%%%%%%%%%%%%%%%%%%%%%%%%%%%%%%%%%%%%%%%%%%%%%%%%%%%%%%%%%%%%%%%%%%%%%%%%%%%%%%%%%%%%%%%%%%%%%%%%%%%%%%%%%%%%%%%%%%%%%%%%%%%%%%%%%%%%%%%%%%%%%%%%%%%%%%%%%%%%%%%%%%%%%%%%%%%%%%%%%%%%%%%%%%%%%%%%%%%%%%%%%%%%%%%%%%%%%%%%%%%%%%%
%TCIDATA{OutputFilter=LATEX.DLL}
%TCIDATA{Version=5.00.0.2552}
%TCIDATA{<META NAME="SaveForMode" CONTENT="1">}
%TCIDATA{Created=Sunday, September 20, 2015 07:35:41}
%TCIDATA{LastRevised=Sunday, September 20, 2015 23:57:12}
%TCIDATA{<META NAME="GraphicsSave" CONTENT="32">}
%TCIDATA{<META NAME="DocumentShell" CONTENT="Standard LaTeX\Blank - Standard LaTeX Article">}
%TCIDATA{CSTFile=40 LaTeX article.cst}

\newtheorem{theorem}{Theorem}
\newtheorem{acknowledgement}[theorem]{Acknowledgement}
\newtheorem{algorithm}[theorem]{Algorithm}
\newtheorem{axiom}[theorem]{Axiom}
\newtheorem{case}[theorem]{Case}
\newtheorem{claim}[theorem]{Claim}
\newtheorem{conclusion}[theorem]{Conclusion}
\newtheorem{condition}[theorem]{Condition}
\newtheorem{conjecture}[theorem]{Conjecture}
\newtheorem{corollary}[theorem]{Corollary}
\newtheorem{criterion}[theorem]{Criterion}
\newtheorem{definition}[theorem]{Definition}
\newtheorem{example}[theorem]{Example}
\newtheorem{exercise}[theorem]{Exercise}
\newtheorem{lemma}[theorem]{Lemma}
\newtheorem{notation}[theorem]{Notation}
\newtheorem{problem}[theorem]{Problem}
\newtheorem{proposition}[theorem]{Proposition}
\newtheorem{remark}[theorem]{Remark}
\newtheorem{solution}[theorem]{Solution}
\newtheorem{summary}[theorem]{Summary}
\newenvironment{proof}[1][Proof]{\noindent\textbf{#1.} }{\ \rule{0.5em}{0.5em}}


\begin{document}


1-3解\qquad 由理想气体状态%
方程得$:$

\qquad 初始状态$:P_{1}(2V)=n_{1}RT_{1}$

$\ $瓶1末状态$:P_{2}V=n_{2}RT_{2}$

\qquad 瓶2末状态$:$ $P_{2}V=n_{3}RT_{1}$

\qquad 物质的量守恒关系$:$ $%
n_{1}=n_{2}+n_{3}$

其中$V$ 为烧瓶的体积$%
,T_{1}=300\unit{K},T_{2}=400\unit{K},P_{1}=0.5\times 101325\unit{Pa}%
,n_{1}=0.7\unit{mol}.$解上面四个方%
程可求出$:$

瓶中压力$P_{2}=\frac{2P_{1}T_{2}}{T_{1}+T_{2}}%
=2\times 2895.71\unit{Pa}=57900\unit{Pa},\qquad $瓶1含H$_{2}$%
物质的量$n_{2}=\frac{n_{1}T_{1}}{T_{1}+T_{2}}=0.3%
\unit{mol},$

瓶2含H$_{2}$物质的量$n_{3}=\frac{%
n_{1}T_{2}}{T_{1}+T_{2}}=0.4\unit{mol}.$

1-5 解\qquad 此时水蒸气的气%
压为\qquad 3168$\unit{Pa}\times 60\%=1900.8\unit{Pa}.$

\qquad 由道尔顿分压定律%
:\qquad $P_{i}V=n_{i}RT\qquad \left( 1,2\right) ,\qquad $其中$%
n_{i}$为各气体的物质的%
量,$V$为所取一部分空气%
的总体积

而$i=O_{2},N_{2}$为书写方便,下%
面将$i$分别记为1,2。

且有\qquad $P_{1}+P_{2}+P_{3}=P\qquad \left( 3\right) ,$其%
中\qquad $P_{3}=1900.8\unit{Pa},P=101325Pa.$

由Amagat's Law,$P(v_{i}V)=n_{i}RT,$可推出在%
混合气体中各气体体%
积比与物质的量之比%
相同$,$故对题目中$O_{2}$和$%
N_{2}$而言

$\frac{v_{1}}{v_{2}}=\frac{n_{1}}{n_{2}}\qquad \left( 4\right) ,$

由$\left( 1,2,4\right) $可得$\frac{P_{1}}{P_{2}}=\frac{%
n_{1}}{n_{2}}=\frac{79}{21}\left( 5\right) ,$由$\left( 3,5\right) $%
联立求得\qquad $P_{1}=78545.1\unit{Pa}$,$P_{2}$=$%
20879.1\unit{Pa}$.

而$\rho =\frac{m_{total}}{V}=M_{1}\frac{n_{1}}{V}+M_{2}\frac{n_{2}}{V}=%
\frac{M_{1}P_{1}+M_{2}P_{2}}{RT}=\frac{28\unit{g}/\unit{mol}\ast 78545.1%
\unit{Pa}+32\unit{g}/\unit{mol}\ast 20879.1\unit{Pa}}{8.314\unit{J}/\left( 
\unit{mol}\cdot \unit{K}\right) \ast 298.15\unit{K}}=\allowbreak
1156.\,\allowbreak 8\unit{g}/\unit{m}^{3}$

$=1.16\times 10^{3}\unit{kg}/\unit{m}^{3}$

1.6\qquad 证明:

在\qquad $PV=nRT\qquad $等式中,固定$T,$%
两边对$V$求导,可得

$\left( \frac{\partial P}{\partial V}\right) _{T}V+P=0;\qquad $即%
有\qquad $\left( \frac{\partial P}{\partial V}\right) _{T}=-\frac{P}{V}%
;$

同理,固定$P$,等式两边对%
$T$求导,可得

$P\left( \frac{\partial V}{\partial T}\right) _{P}=nR;\qquad $即%
有\qquad $\left( \frac{\partial V}{\partial T}\right) _{P}=\frac{nR}{P}%
;$

固定$V$,等式两边对$T$求%
导,可得

$\left( \frac{\partial P}{\partial T}\right) _{V}V=nR;\qquad $即%
有\qquad $\left( \frac{\partial P}{\partial T}\right) _{V}=\frac{nR}{V}%
;$

所以\qquad $\left( \frac{\partial P}{\partial V}\right) _{T}=-%
\frac{\left( \frac{\partial P}{\partial T}\right) _{V}}{\left( \frac{%
\partial V}{\partial T}\right) _{P}}.$

1.7\qquad $\left( 1\right) $欲证\qquad $\alpha =\kappa \beta P,$%
即证\qquad $\left( \frac{\partial V}{\partial T}\right)
_{P}=-\left( \frac{\partial V}{\partial P}\right) _{T}\left( \frac{\partial P%
}{\partial T}\right) _{V}.$

因为\qquad 气体的$P,V,T$三者%
存在函数关系\qquad $f\left( P,V,T\right)
=0.$

固定$P$,等式两边对$T$求%
导,可得\qquad $\frac{\partial f}{\partial V}\left( \frac{%
\partial V}{\partial T}\right) _{P}+\frac{\partial f}{\partial T}=0;\qquad $%
即有\qquad $\left( \frac{\partial V}{\partial T}\right) _{P}=-%
\frac{\frac{\partial f}{\partial T}}{\frac{\partial f}{\partial V}}\qquad
\left( 1\right) ;$

固定$T$,等式两边对$P$求%
导,可得$\frac{\partial f}{\partial P}+\frac{\partial f}{%
\partial V}\left( \frac{\partial V}{\partial P}\right) _{T}=0;\qquad $%
即有\qquad $\left( \frac{\partial V}{\partial P}\right) _{T}=-%
\frac{\frac{\partial f}{\partial P}}{\frac{\partial f}{\partial V}}\qquad
\left( 2\right) ;$

固定$V,$两边对$T$求导可%
得\qquad $\frac{\partial f}{\partial P}\left( \frac{\partial P}{%
\partial T}\right) _{V}+\frac{\partial f}{\partial T}=0,$即有%
\qquad $\left( \frac{\partial P}{\partial T}\right) _{V}=-\frac{\frac{%
\partial f}{\partial T}}{\frac{\partial f}{\partial P}}\qquad \left(
3\right) ;$

由$\left( 1,2,3\right) $得\qquad \qquad $\left( \frac{\partial V%
}{\partial T}\right) _{P}=-\left( \frac{\partial V}{\partial P}\right)
_{T}\left( \frac{\partial P}{\partial T}\right) _{V}.$

1.13\qquad 气压计读出水银所%
产生的液压,设两次读%
数外界压强分别为$%
P_{e1},P_{e2}\qquad $两次读数时的液%
压分别为\qquad $P_{r1},P_{r2}$

而空气柱的长度分别%
为$l_{1},l_{2}$则由Pascal's Law 及液体%
压强公式可列出方程%
为

$\left( P_{e1}-P_{r1}\right) Sl_{1}=\left( P_{e2}-P_{r2}\right) Sl_{2}\qquad
\left( 1\right) $

$\left\vert P_{r1}-P_{r2}\right\vert =\left\vert l_{1}-l_{2}\right\vert \rho
g\qquad \left( 2\right) \qquad $其中$\rho $为水%
银的密度等于$13.5\unit{g}/\unit{cm}^{3}.$

由$\left( 1,2\right) $解得\qquad $l_{1}=\frac{\left(
P_{e2}-P_{r2}\right) \left( P_{r1}-P_{r2}\right) }{\left(
P_{e1}-P_{e2}+P_{r2}-P_{r1}\right) \rho g}=\frac{\left( 98.66\unit{kPa}-98.13%
\unit{kPa}\right) \times \left( 99.73\unit{kPa}-98.13\unit{kPa}\right) }{%
\left( 100.66\unit{kPa}-98.66\unit{kPa}+98.13\unit{kPa}-99.73\unit{kPa}%
\right) \times 13.5\unit{g}/\unit{cm}^{3}\times 9.8\times 10^{2}\unit{cm}/%
\unit{s}^{2}}$

$=\allowbreak 1.\,\allowbreak 602\,4\times 10^{-4}\frac{\unit{s}^{2}\unit{cm}%
^{2}\unit{kPa}\allowbreak }{\unit{g}}\approx 1.6\unit{cm}$

故所求空间为\qquad $l_{1}S=1.6\unit{cm}%
^{3}$

\end{document}
