
\documentclass{ctexart}
%%%%%%%%%%%%%%%%%%%%%%%%%%%%%%%%%%%%%%%%%%%%%%%%%%%%%%%%%%%%%%%%%%%%%%%%%%%%%%%%%%%%%%%%%%%%%%%%%%%%%%%%%%%%%%%%%%%%%%%%%%%%%%%%%%%%%%%%%%%%%%%%%%%%%%%%%%%%%%%%%%%%%%%%%%%%%%%%%%%%%%%%%%%%%%%%%%%%%%%%%%%%%%%%%%%%%%%%%%%%%%%%%%%%%%%%%%%%%%%%%%%%%%%%%%%%
\usepackage{amsmath}

\setcounter{MaxMatrixCols}{10}
%TCIDATA{OutputFilter=LATEX.DLL}
%TCIDATA{Version=5.00.0.2552}
%TCIDATA{<META NAME="SaveForMode" CONTENT="1">}
%TCIDATA{Created=Thursday, November 12, 2015 17:20:37}
%TCIDATA{LastRevised=Sunday, November 22, 2015 22:19:55}
%TCIDATA{<META NAME="GraphicsSave" CONTENT="32">}
%TCIDATA{<META NAME="DocumentShell" CONTENT="Scientific Notebook\Blank Document">}
%TCIDATA{CSTFile=Math with theorems suppressed.cst}
%TCIDATA{PageSetup=72,72,72,72,0}
%TCIDATA{AllPages=
%F=36,\PARA{038<p type="texpara" tag="Body Text" >\hfill \thepage}
%}


\newtheorem{theorem}{Theorem}
\newtheorem{acknowledgement}[theorem]{Acknowledgement}
\newtheorem{algorithm}[theorem]{Algorithm}
\newtheorem{axiom}[theorem]{Axiom}
\newtheorem{case}[theorem]{Case}
\newtheorem{claim}[theorem]{Claim}
\newtheorem{conclusion}[theorem]{Conclusion}
\newtheorem{condition}[theorem]{Condition}
\newtheorem{conjecture}[theorem]{Conjecture}
\newtheorem{corollary}[theorem]{Corollary}
\newtheorem{criterion}[theorem]{Criterion}
\newtheorem{definition}[theorem]{Definition}
\newtheorem{example}[theorem]{Example}
\newtheorem{exercise}[theorem]{Exercise}
\newtheorem{lemma}[theorem]{Lemma}
\newtheorem{notation}[theorem]{Notation}
\newtheorem{problem}[theorem]{Problem}
\newtheorem{proposition}[theorem]{Proposition}
\newtheorem{remark}[theorem]{Remark}
\newtheorem{solution}[theorem]{Solution}
\newtheorem{summary}[theorem]{Summary}
\newenvironment{proof}[1][Proof]{\noindent\textbf{#1.} }{\ \rule{0.5em}{0.5em}}


\begin{document}


\bigskip \bigskip 物化第9周作业\qquad 
赵丰\qquad 2013012178

3.14$\left( 1\right) $ 该相变过程在%
等温等压无非体积功%
条件下进行$\implies \Delta G=0,$忽%
略液体的体积$\implies \Delta
A=-W=-P_{ext}V_{g}=-nRT=-2935.\,\allowbreak 5\unit{J}.$计算%
结果表明,该反应可逆.

$\left( 2,3\right) $由于反应前后体%
积压力均不相同,故根%
据计算结果不能由Helmholz or
Gibbs Criteria 判断反应的方向. $%
\left( 2,3\right) $均为不可逆过程,%
为求状态函数变,设计%
如下可逆过程II和III,并%
设过程末态压力为P, 由$%
\left( 1\right) $的计算结果$\Delta
G_{II}=0,\Delta A_{II}=-2935.\,\allowbreak 5\unit{J}.$

\FRAME{ftbpF}{2.4846in}{0.9003in}{0pt}{}{}{Figure}{\special{language
"Scientific Word";type "GRAPHIC";display "USEDEF";valid_file "T";width
2.4846in;height 0.9003in;depth 0pt;original-width 5.4371in;original-height
2.8331in;cropleft "0";croptop "1";cropright "1";cropbottom "0";tempfilename
'NXP83J03.wmf';tempfile-properties "XPR";}}

对III,为理想气体等温压%
缩或膨胀,$\Delta G_{III}=\Delta A_{III}=-W=nRT\ln 
\frac{P_{1}}{P_{2}}$

$\left( 2\right) \Delta A=\Delta A_{II}+\Delta A_{III}=-2935.\,5\unit{J}%
+8.31\times \left( 273.15+80.1\right) \times \ln \frac{91193}{101325}\unit{J}
$

=$-3244.\,\allowbreak 8\unit{J},\Delta G=\Delta G_{II}+\Delta
G_{III}=-309.\,\allowbreak 27\unit{J}.$

$\left( 3\right) $同理$\allowbreak \Delta G=\allowbreak
279.\,\allowbreak 80\unit{J},\Delta A=\allowbreak -2655.\,\allowbreak 7\unit{%
J}$

$3-39$由Kirchhoff 公式,对80\symbol{126}300K任%
一温度$T$,

$\Delta H_{m}\left( T\right) =\Delta H_{m}\left( 273K\right)
+\int_{273}^{T}\left( C_{p,m}\left( \text{白,Sn}\right) -C_{p,m}\left( 
\text{黑,Sn}\right) \right) dT$

$=2226+\int_{273}^{T}\left( 2.05+13.6\times 10^{-8}\left( 300-x\right)
^{3}\right) dx$

$=\allowbreak 5.\,\allowbreak 722T-0.018\,36T^{2}+4.\,\allowbreak 08\times
10^{-5}T^{3}-3.\,\allowbreak 4\times 10^{-8}\allowbreak
T^{4}+1391.\,\allowbreak 0$

\bigskip $\Delta H_{m}\left( T\right) $单位为$\unit{J}%
\cdot \unit{mol}^{-1}$

292K温度下,反应达到平衡%
$\implies \Delta G\left( 292K\right) =0,$

以此温度为参照,由%
Gibbs-Helmholtz公式:

$\frac{\Delta G_{m}\left( T\right) }{T}=-\int_{292}^{T}\frac{5.\,\allowbreak
722x-0.018\,36x^{2}+4.\,\allowbreak 08\times 10^{-5}x^{3}-3.\,\allowbreak
4\times 10^{-8}\allowbreak x^{4}+1391.\,\allowbreak 0}{x^{2}}dx$

$=\allowbreak 0.018\,36T-5.\,\allowbreak 722\ln T+5.\,\allowbreak 722\ln
292+\allowbreak \frac{1391.0}{T}-2.\,\allowbreak 04\times 10^{-5}T^{2}$

$+1.\,\allowbreak 133\,3\times 10^{-8}\allowbreak T^{3}-8.\,\allowbreak
667\,6$

$\implies \Delta G_{m}\left( T\right) =-5.\,\allowbreak 722\allowbreak T\ln
T+\allowbreak 23.\,\allowbreak 815T+0.018\,36T^{2}-2.\,\allowbreak 04\times
10^{-5}T^{3}$

$+1.\,\allowbreak 133\,3\times 10^{-8}T^{4}+\allowbreak 1391.0$

$\Delta G_{m}\left( T\right) $单位为$\unit{J}\cdot \unit{%
mol}^{-1}$

Atkins:

5-11 Show that $\left( \frac{\partial H}{\partial V}\right)
_{T}=-V^{2}\left( \frac{\partial P}{\partial T}\right) _{V}\left( \frac{%
\partial \left( T/V\right) }{\partial V}\right) _{P}.$

$dH=VdP+TdS,$温度不变两边同时%
除以$dV\implies $

$\left( \frac{\partial H}{\partial V}\right) _{T}=V\left( \frac{\partial P}{%
\partial V}\right) _{T}+T\left( \frac{\partial S}{\partial V}\right) _{T},$%
by Maxwell's relation $\left( \frac{\partial S}{\partial V}\right)
_{T}=\left( \frac{\partial P}{\partial T}\right) _{V}$

$\implies \left( \frac{\partial H}{\partial V}\right) _{T}=V\left( \frac{%
\partial P}{\partial V}\right) _{T}+T\left( \frac{\partial P}{\partial T}%
\right) _{V},$

等式右边将$\left( \frac{\partial \left(
T/V\right) }{\partial V}\right) _{P}$展开有:

\bigskip RightHand$=-V\left( \frac{\partial P}{\partial T}\right) _{V}\left( 
\frac{\partial T}{\partial V}\right) _{P}+T\left( \frac{\partial P}{\partial
T}\right) _{V},$应用链关系

$\left( \frac{\partial P}{\partial T}\right) _{V}\left( \frac{\partial T}{%
\partial V}\right) _{P}\left( \frac{\partial V}{\partial P}\right)
_{T}=-1\implies $RightHand$=V\left( \frac{\partial V}{\partial P}\right)
_{T}+T\left( \frac{\partial P}{\partial T}\right) _{V}=$LeftHand

D因此原热力学关系式%
得证.

6-19$\left( a\right) $

\FRAME{itbpF}{3.0104in}{1.6812in}{0in}{}{}{Figure}{\special{language
"Scientific Word";type "GRAPHIC";display "USEDEF";valid_file "T";width
3.0104in;height 1.6812in;depth 0in;original-width 5.6879in;original-height
3.4999in;cropleft "0";croptop "1";cropright "1";cropbottom "0";tempfilename
'NXPEDF05.wmf';tempfile-properties "XPR";}}\FRAME{itbpF}{2.9533in}{1.6501in}{%
0in}{}{}{Figure}{\special{language "Scientific Word";type "GRAPHIC";display
"USEDEF";valid_file "T";width 2.9533in;height 1.6501in;depth
0in;original-width 5.7398in;original-height 3.5622in;cropleft "0";croptop
"1";cropright "1";cropbottom "0";tempfilename
'NXPEOK06.wmf';tempfile-properties "XPR";}}

$\left( b\right) $令$P_{s-l}\left( T\right) =1$解得\qquad $%
T=178.182\unit{K},$即为standard melting point of toluene.

令$\ln P_{l-v}\left( T\right) =0$解得\qquad $T=383.612%
\unit{K},$即为standard boiling point of toluene.

$\left( c\right) \qquad $First we calculate $\frac{dP}{dT}$at $T=383.612%
\unit{K},$which gives 0.0285$\unit{bar}/\unit{K}$

From the Clapeyron equation $\frac{dP}{dT}=\frac{\Delta _{vap}H}{T\Delta
_{vap}V}\implies $ $\Delta _{vap}H=33.4\unit{kJ}/\unit{mol}$

4-1 $\left( 1\right) $ $\frac{6!}{3!2!1!}=\allowbreak 60.$

$\left( 2\right) 60\times 3^{3}\times 2^{2}\times 4=\allowbreak 25\,920$

4-4 For $j=2$ and $j=3,\Delta \epsilon _{r}=\frac{6h^{2}}{8\pi ^{2}I}=\frac{%
6\times \left( 6.6262\times 10^{-34}\right) ^{2}}{8\pi ^{2}\times 10\times
10^{-47}}=\allowbreak 3.\,\allowbreak 336\,5\times 10^{-22}\unit{J}$

\end{document}
